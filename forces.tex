\section{Force models}\label{sec:forces}
In this section, we list the force densities associated with each of the energy densities
given in Section~\ref{sec:energy}. For simplicity, we adopt the Einstein summation
notation. A Greek letter featuring as a subscript and superscript within a term indicates
summation over ${1,\,2}$ for that letter. We also adopt the comma notation for partial
differentiation, where subscripts following a comma indicate partial differentiation with
respect to the corresponding coordinates, \latin{e.g}.,
$\phi_{,\alpha\beta} = \partial^2\phi/\partial\q[\alpha]\partial\q[\beta]$.

We begin with Hookean and damped spring forces, which are the simplest of the forces we
consider. The spring force density takes the form
\begin{equation}
    \F_\text{spring} = -k (\X - \X') - \eta(\vec{\dot{X}}-\vec{\dot{X}}').
\end{equation}
where $\X'$ is the tether location, $\vec{\dot{X}}$ is the surface velocity,
$\vec{\dot{X}}'$ is the prescribed velocity, $k$ is the spring constant, and $\eta$ is
the damping constant.  It is common practice to treat each spring as an individual, so
that the quadrature weight $\weight[j]$ is absorbed into the coefficients, and $k$ has
units of force per length and $\eta$ units of force-time per length. Implementation of
these forces requires no geometric information outside of positions and velocities.

Next, we consider tension, which generates forces based on stretching or compressing
and relative direction of the surface tangent vectors. We construct the \emph{metric
tensor}, $\metric_{\alpha\beta} = \X_{,\alpha}\cdot\X_{,\beta}$, which encodes local
information about distance and area. Its inverse, denoted by $\metric^{\alpha\beta}$, is
called the dual metric. Likewise, we construct a metric tensor for the reference
configuration, $\reference\metric_{\alpha\beta}$, and its dual. The invariants of the
Green-Lagrange strain tensor,
\begin{equation}
    \epsilon_\alpha^\beta = \frac12\left(\metric_{\alpha\mu}\reference\metric^{\mu\beta}-\Kronecker_\alpha^\beta\right),
\end{equation}
encode information about relative changes in lengths and areas. We can then use this to
express the Tension energy density invariants $I_1 = 2\epsilon_\mu^\mu$ and
$I_2 = 4\epsilon + I_1$, where $\epsilon_\mu^\mu$ and $\epsilon$ are the trace and
determinant of $\epsilon_\alpha^{\smash\beta}$, respectively. The tension energy density
function, $W(I_1,\,I_2)$, is therefore a function only of the tangents, $\X_{,\alpha}$.
For Skalak's Law~\eqref{eq:skalak-law} and neo-Hookean tension~\eqref{eq:neohookean}, the
tension force density is computed~\cite{Maxian:2018ek} via
\begin{equation}\label{eq:tension-force}
    \F_\text{tension} = \frac{1}{\sqrt{\reference\metric}}\left(\sqrt{\reference\metric}s^{\alpha\beta}\X_{,\beta}\right)_{,\alpha},
\end{equation}
where the second Piola-Kirchhoff stress tensor, $s^{\alpha\beta}$, is given by
\begin{equation}
    s^{\alpha\beta} = 2\frac{\partial W}{\partial I_1} \hat{g}^{\alpha\beta} + 2I_2\frac{\partial W}{\partial I_2} g^{\alpha\beta}.
\end{equation}
Because the tension force density is expressed in relation to the reference
configuration, the force is computed by multiplying by quadrature weights for the
reference configuration, which do not change over the course of a simulation.

To compute bending force density, we first construct the unit normal vector by taking the
cross product of the tangent vectors and normalizing:
\begin{equation}\label{eq:unit-normal}
    \n = \frac{1}{\sqrt{\metric}} (\X_{,1}\times\X_{,2}).
\end{equation}
The tensor $b_{\alpha\beta} = \n\cdot\X_{,\alpha\beta}$ encodes information about the
curvature of the surface. The principle curvatures are the eigenvalues of the
\emph{shape tensor}, $K_\alpha^\beta = b_{\alpha\mu}\metric^{\mu\beta}$, with
$2H = K_\mu^\mu$ the total curvature, and $K$ the Gaussian curvature. The Canham-Helfrich
force density takes the form~\cite{Zhongcan:1989ue}%
\begin{equation}\label{eq:bending-force}
    \F_\text{CH} = -4\kappa\left(\laplacian(H-H')+2(H-H')(H^2-K+HH')\right)\n,
\end{equation}
where $\laplacian$ is the Laplace-Beltrami operator. We can compute $H$ and $K$ using the
formulas above, but $\laplacian H$ requires up to fourth derivatives of $\X$. In lieu of
computing higher-order derivatives, we compute $H$ and apply the discrete
Laplace-Beltrami operator.

Finally, we consider dissipative forces, which cause the membrane to exhibit a
viscoelastic response to strain. With surface velocity $\vec{\dot{X}}$, the metric tensor
changes in time according to 
\begin{equation}
    \dot{\metric}_{\alpha\beta} = \vec{\dot{X}}_{,\alpha}\cdot\X_{,\beta} + \X_{,\alpha}\cdot\vec{\dot{X}}_{,\beta}.
\end{equation}
The dissipative force density takes the form~\cite{Rangamani:2012hi}
\begin{equation}\label{eq:dissip-force}
    \F_\text{dissip} = \frac{\nu}{\sqrt{\metric}}\left(\sqrt{\metric}\metric^{\alpha\mu}\dot{\metric}_{\mu\lambda}\metric^{\lambda\beta}\X_{,\beta}\right)_{,\alpha},
\end{equation}
where $\nu$ is the membrane viscosity.

Rewriting~\eqref{eq:tension-force},~\eqref{eq:dissip-force}, and the Laplace-Beltrami
operator in~\eqref{eq:bending-force} in terms of first and second derivatives with
respect to parametric variables,
\begin{equation}\label{eq:expanded-op}
    \frac1{\sqrt{\metric}}\left(\sqrt{\metric}D^\alpha_\mu \metric^{\mu\beta}\phi_{,\beta}\right)_{,\alpha}
    = \left[\left(D^\alpha_\mu \metric^{\mu\beta}\right)_{,\alpha} + \frac{\metric_{,\alpha}}{2\metric}D^\alpha_\mu \metric^{\mu\beta}\right]\phi_{,\beta}+D^\alpha_\mu \metric^{\mu\beta}\phi_{,\alpha\beta},
\end{equation}
with $D^\alpha_\mu = s^{\alpha\beta}\metric_{\beta\mu}$,
$D^\alpha_\mu = \nu \metric^{\alpha\beta}\dot{g}_{\beta\mu}$, and
$D^\alpha_\mu = \delta^\alpha_\mu$,
respectively, we find that we require only discrete first and second derivative operators
to compute a wide variety of forces.

% vim: cc=90 tw=89
