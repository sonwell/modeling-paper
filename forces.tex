\section{Force models}\label{sec:forces}
In this section, we list the force densities associated with each of the energy densities
given in Section~\ref{sec:energy}. For simplicity, we adopt the Einstein summation
notation. A Greek letter featuring as a subscript and superscript within a term,
\latin{e.g.}, $a_\alpha b^\alpha$, indicates summation over $\{1,\,2\}$ for that letter.
We also adopt the comma notation for partial differentiation, where subscripts following
a comma indicate partial differentiation with respect to the corresponding coordinates,
\latin{e.g}., $\phi_{,\alpha\beta} = \partial^2\phi/\partial\q[\alpha]\partial\q[\beta]$.
The surface coordinates $\q[1]$ and $\q[2]$ correspond to $\theta$ and $\phi$ of
Sections~\ref{sec:energy} and~\ref{sec:rbfs} in either order.

We begin with Hookean and damped spring forces, whose force density takes the form
\begin{equation}
    \F_\text{spring} = -k (\X - \X') - \eta(\vec{\dot{X}}-\vec{\dot{X}}').
\end{equation}
where $\X'$ is the tether location, $\vec{\dot{X}}$ is the surface velocity,
$\vec{\dot{X}}'$ is the prescribed velocity, $k$ is the spring constant, and $\eta$ is
the damping constant. It is common practice to treat each spring individually, so that
the quadrature weight $\weight[j]$ is absorbed into the coefficients: $k$ has units of
force per length and $\eta$ units of force-time per length. Implementation of these
forces requires no geometric information outside of positions and velocities. These 
are only used for the endothelium.

Next, we consider tension, which generates forces based on stretching or compressing of the elastic surface. This is somewhat more complicated than the Hookean spring case. First, we define the \emph{metric
tensor}, $\metric_{\alpha\beta} = \X_{,\alpha}\cdot\X_{,\beta}$  (which encodes local
information about distance and area) and its inverse, $\metric^{\alpha\beta}$ (sometimes called the dual metric). We similarly define the metric tensor for the reference
configuration, $\reference\metric_{\alpha\beta}$, and its dual. This allows us to write the Green-Lagrange strain tensor as
\begin{equation}
    \epsilon_\alpha^\beta = \frac12\left(\metric_{\alpha\mu}\reference\metric^{\mu\beta}-\Kronecker_\alpha^\beta\right),
\end{equation}
The invariants of this tensor encode information about relative changes in lengths and areas, and therefore can be used to write a tension force density. The invariants can be expressed as
\begin{align}
I_1 &= 2\epsilon_\mu^\mu, \\ 
I_2 &= 4\epsilon + I_1,
\end{align}
where $\epsilon_\mu^\mu$ and $\epsilon$ are the trace and determinant, respectively, of $\epsilon_\alpha^{\smash\beta}$. For Skalak's Law~%
\eqref{eq:skalak-law} and neo-Hookean tension~\eqref{eq:neohookean}, we first define the second Piola-Kirchhoff stress tensor using the invariants $I_1$ and $I_2$ as
\begin{equation}
s^{\alpha\beta} = 2\frac{\partial W}{\partial I_1} \hat{g}^{\alpha\beta} + 2I_2\frac{\partial W}{\partial I_2} g^{\alpha\beta},
\end{equation}
where $W(I_1,I_2)$ is the tension energy density function (see Section 4). This in turn allows us to define the tension force density~\cite{Maxian:2018ek}
\begin{equation}\label{eq:tension-force}
    \F_\text{tension} = \frac{1}{\sqrt{\reference\metric}}\left(\sqrt{\reference\metric}s^{\alpha\beta}\X_{,\beta}\right)_{,\alpha},
\end{equation}
where $\X_{,\beta}$ refer to the tangent vectors on the surface. Because the tension force density is expressed in relation to the reference
configuration, the force is computed by multiplying by quadrature weights for the
reference configuration, which do not change over the course of a simulation.

Our models also contain terms to penalize bending. First, given tangent vectors $\X_{,1}$ and $\X_{,2}$, the unit normal vector to the surface is given as
\begin{equation}\label{eq:unit-normal}
\n = \frac{1}{\sqrt{\metric}} (\X_{,1}\times\X_{,2}).
\end{equation}
This in turn allows us to define the symmetric tensor $b_{\alpha\beta} = \n\cdot\X_{,\alpha\beta}$, which contains the coefficients of the second fundamental form, and the \emph{shape tensor} $K_\alpha^\beta = b_{\alpha\mu}\metric^{\mu\beta}$. The principal curvatures are the eigenvalues of the shape tensor. More importantly, the trace of this tensor is twice the mean curvature $2H = K_\mu^\mu$, and its determinant $K$ is the Gaussian curvature. We use $H$ and $K$ within a standard expression for the Canham-Helfrich force density~\cite{Zhongcan:1989ue} to obtain the following bending force density:
\begin{equation}\label{eq:bending-force}
\F_\text{CH} = -4\kappa\left(\laplacian(H-H')+2(H-H')(H^2-K+HH')\right)\n,
\end{equation}
where $\laplacian$ is the Laplace-Beltrami operator. We can compute $H$ and $K$ using the
formulas above, but $\laplacian H$ requires up to fourth derivatives of $\X$. In numerical simulations, we compute $H$ pointwise and apply the discrete
Laplace-Beltrami operator to obtain this $\laplacian H$.

Finally, we consider dissipative forces, which cause the membrane to exhibit a
viscoelastic response to strain. With surface velocity $\vec{\dot{X}}$, the metric tensor
changes in time according to 
\begin{equation}
    \dot{\metric}_{\alpha\beta} = \vec{\dot{X}}_{,\alpha}\cdot\X_{,\beta} + \X_{,\alpha}\cdot\vec{\dot{X}}_{,\beta}.
\end{equation}
The dissipative force density takes the form~\cite{Rangamani:2012hi}
\begin{equation}\label{eq:dissip-force}
    \F_\text{dissip} = \frac{\nu}{\sqrt{\metric}}\left(\sqrt{\metric}\metric^{\alpha\mu}\dot{\metric}_{\mu\lambda}\metric^{\lambda\beta}\X_{,\beta}\right)_{,\alpha},
\end{equation}
where $\nu$ is the membrane viscosity. 

In general, for all force calculations, it is possible to rewrite ~\eqref{eq:tension-force},~\eqref{eq:dissip-force}, and the Laplace-Beltrami
operator in~\eqref{eq:bending-force} in terms of first and second derivatives with
respect to parametric variables, thereby ensuring that we require only discrete first and second derivative operators to compute a wide variety of forces.

% vim: cc=90 tw=89
