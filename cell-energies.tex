\section{Cell energy and power models}\label{sec:energy}

In this section, we describe the various forms of energy density (energy per area) and power density (power per
area) used in our simulations, and give analytic expressions for each. The corresponding force densities are given
in~\ref{sec:forces}. We use five kinds of densities: spring energy, damped spring power, tension energy,
dissipative power, and Canham-Helfrich bending energy. For constitutive law $W$, we define the functional
\begin{equation}
    \energy[\X, \U] = \int\limits_\interface W(\X, \U, \ldots) \d\X,
\end{equation}
where the ellipsis indicates that $W$ may depend on spatial derivatives of $\X$ or $\U$. The force density
associated with $W$ is found by computing the first variation of $\energy$,
\begin{equation}
    \F = -\delta\energy,
\end{equation}
with respect to $\X$ for energy densities and to $\U$ for power densities. Because our ultimate goal is a
three-dimensional simulation, we limit our descriptions to the three-dimensional case. Considerations for the
two-dimensional case are treated elsewhere~\cite{Peskin:2002go,Erickson:2010uzba}.

We begin with Hookean energy and damped spring power density. These have the simplest constitutive laws we use,
depend only on surface locations and surface velocities, and take the form
\begin{equation}
        W_\text{Hk}(\X) = \frac{k}2 {\|\X - \X'\|}^2\quad\text{and}\quad
        W_\text{damped}(\U) = \frac{\eta}2{\|\U - \U'\|}^2,
\end{equation}
where $\X'=\X'(\theta, \varphi, t)$ is the tether location for $\X(\theta, \varphi, t)$,
$\U'=\U'(\theta, \varphi, t)$ is the prescribed velocity of the tether point, $k$ is the spring constant, and
$\eta$ is the damping constant.  Due to the lack of information about the mechanical properties of endothelial
cells, we model the endothelium as a rigid, stationary object with $\ethm{k}=2.5\dynpercm$ and
$\ethm\eta=2.5\sci{-7}\dynsecpercm$, chosen to be as large as possible for the chosen spatial and temporal step
size with prescribed velocity $\U' = \vec{0}$. We compare different choices for $\X'$ in \cref{sec:whole-blood}.

Next, we consider the tension energy densities for RBCs and platelets. These penalize stretching and areal
dilation of the cell membranes. Let $\lambda_1$ and $\lambda_2$ be the principal extensions, \latin{i.e.}, the
maximal and minimal ratios of stretching relative to a reference configuration. We define the invariants
$I_1=\lambda_1^2+\lambda_2^2-2$ and $I_2 = \lambda_1^2\lambda_2^2-1$, which measure relative changes in length and
area, respectively, such that $I_1 = I_2 = 0$ correspond to a rigid body motion. We express the tension density in
terms of these invariants. Skalak's Law was designed specifically for RBCs~\cite{Skalak:1973tp}:
\begin{equation}\label{eq:skalak-law}
    W_\text{Sk}(I_1, I_2) = \frac{E}4\left(I_1^2 + 2I_1 - 2I_2\right) + \frac{G}4 I_2^2.
\end{equation}
$E$ is the shear modulus, and $G$ is the bulk modulus. The shear and bulk moduli for RBCs are estimated to be
$E = 6\sci{-3}\dynpercm$ and $G = 5\sci2\dynpercm$~\cite{Mohandas:1994tg}, but we follow Fai \latin{et al.}~%
\cite{Fai:2013do} and use $\rbc{E} = 2.5\sci{-3}\dynpercm$ and $\rbc{G} = 2.5\sci{-1}\dynpercm$. We use
the shape given by Evans \& Fung~\cite{Evans:1972uf} for the reference RBC with radius $R_0 = 3.91\um$,
\begin{equation*}
    \rbc{\vec{\hat{X}}}(\theta, \varphi) = R_0\begin{bmatrix}
            \cos\theta\cos\varphi \\
            \sin\theta\cos\varphi \\
            z(\cos\varphi)\sin\varphi
    \end{bmatrix},
\end{equation*}
where $(\theta, \varphi)\in(-\pi, \pi]\times[-\pi/2, \pi/2]$ and $z(r) = 0.105 + r^2 - 0.56r^4$. Platelets, on the
other hand, do not have a purpose-built constitutive law, but are stiffer than RBCs. We use the neo-Hookean model
\begin{equation}\label{eq:neohookean}
    W_\text{nH}(I_1, I_2) = \frac{E}2\left(\frac{I_1+2}{\sqrt{I_2+1}}-2\right) + \frac{G}2 {\left(\sqrt{I_2+1}-1\right)}^2
\end{equation}
with $\plt{E} = 1\sci{-1}\dynpercm$ and $\plt{G} = 1\dynpercm$, and an ellipsoidal reference configuration~%
\cite{Frojmovic:1982wk}
\begin{equation*}
    \plt{\hat{\vec{X}}}(\theta, \varphi) = \begin{bmatrix}
            1.55\um\cos\theta\cos\varphi \\
            1.55\um\sin\theta\cos\varphi \\
            0.5\um\sin\varphi
    \end{bmatrix}.
\end{equation*}

Platelets and RBCs also respond to changes in membrane curvature. Let $H$ be the membrane's mean curvature. The
Canham-Helfrich bending energy density takes the form~\cite{Canham:1970wx}
\begin{equation}\label{eq:bending-energy}
    W_\text{CH}(H) = 2\kappa {(H-H')}^2,
\end{equation}
where $\kappa$ is the bending modulus in units of energy, and $H'$ is the spontaneous or \emph{preferred}
curvature. An RBC generates a relatively weak response to changes in its curvature. Its bending modulus is
estimated to be in the range $0.3$--$4\sci{-12}\erg$~\cite{Mohandas:1994tg}. We use a bending modulus of
$\rbc\kappa = 2\sci{-12}\erg$ and a preferred curvature $H' = 0$ for RBCs. RBCs, therefore, tend to locally
flatten their membranes. For platelets, we use a larger bending modulus of $\plt\kappa=2\sci{-11}\erg$
and a preference for its reference curvature. Together with the neo-Hookean tension above, this maintains a fairly
rigid platelet.

Finally, we consider dissipative power, which causes the membrane to exhibit a viscoelastic response to strain. It
takes the form~\cite{Rangamani:2012hi}
\begin{equation}\label{eq:dissip-energy}
    W_\text{dissip}(\dot{\lambda}_1, \dot{\lambda}_2) = \frac{\nu}{2}\left(\frac{\dot{\lambda}_1^2}{\lambda_1^2} + \frac{\dot{\lambda}_2^2}{\lambda_2^2}\right),
\end{equation}
where $\nu$ is the membrane viscosity, and $\dot{\lambda}_i$ is the rate of change of $\lambda_i$. We imbue only
the RBC with viscoelasticity. We find this effective in eliminating some numerical instabilities. While Evans \&
Hochmuth suggest a viscosity of approximately $1\sci{-3}\dynsecpercm$~\cite{Evans:1976tx}, we find this to be
prohibitively expensive in practice, due to time step restrictions, and instead use
$\rbc\nu = 2.5\sci{-7}\dynsecpercm$.
