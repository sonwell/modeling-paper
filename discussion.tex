\section{Discussion}\label{sec:conclusion}

We have applied the RBF-IB methodology to the problem of whole blood simulation. RBCs
experience contacts in various configurations. Despite using a different number of data
and sample sites to discretize them, the RBCs remain distinct. We have shown that a
continuous energy RBF-based model RBC extended with dissipative forces exhibits tumbling
and tank-treading behaviors. We used model RBCs, platelets, and a model endothelium to
simulate the flow of whole blood over a perturbed vessel wall. We considered both flat
and bumpy endothelial shapes and flow with and without RBCs along a bumpy wall.

The most prominent result of the whole blood simulations is that simulating platelets,
but neglecting the influence of RBCs, cannot capture the true nature of platelet motion.
Platelets experience augmented shear rates due to RBCs. RBCs confine platelets to the
cell-free layer. Simulations without RBCs exhibit regular platelet tumbling or wobbling.
In reality, the motion of platelets is more chaotic. Interaction with an RBC would almost
certainly disturb a wobbling motion and have been shown to delay tumbling. On a flat
wall, certain platelet behaviors would only be observed by considering numerous starting
configurations.

The effect of RBCs generally overwhelms the effects of the wall. Figure~%
\ref{fig:flow-profiles} shows that the flat wall yields slightly faster fluid velocities,
but the flow profiles for flat and bumpy walls are qualitatively the same, except for the
immediate vicinity of the wall. The result is a region of space between the bumps with a
velocity gradient. Platelets following the shape of the bumpy wall crest the bump, dip
into the region of lower velocity, and tumble. This feature is absent from the flat wall,
but tumbling is not an extraordinary behavior. In either case, we observe unicycling, in
which the platelet rolls in the flow direction along its edge. Unicycling can be
stabilized by RBCs, when RBCs flank the platelet, or by partially encapsulating the
platelet as it flows overhead. It can also be destabilized by RBC passing on one side.
The endothelial protrusions are also sufficient to orient a platelet in the unicycling,
but without RBCs to confine the platelet near the wall, the platelet does not roll along
the wall for long. However, we find that unicycling seems to be stable. Unicycling also
highlights the need for 3-dimensional simulations---it is a behavior that would not be
captured by a 2-dimensional simulation. We also observe platelet-endothelial interactions
for both endothelial shapes. These interactions are typically caused by RBCs driving the
platelet into the endothelium. The collisions are characterized by significant
deformation to the platelet and a speed reduction. From a qualitative standpoint, the
endothelial shape alone has minimal impact on the motion of the platelets, meaning that
for modeling a healthy blood vessel, a flat wall suffices.

We consider an alternative interpretation of the flat wall: a model exposed
subendothelium. Near contact with the subendothelium triggers platelet activation.
Unicycling keeps an edge of the platelet near the wall, without hindering mobility. The
vertical alignment is often maintained much longer than wall contacts from tumbling, and
we do not observe wobbling in the presence of RBCs. We propose unicycling as an effective
means by which platelets survey the vasculature for injury. Of course, the platelet can
only distinguish a healthy vessel from an injury by encountering the necessary chemical
signals. Until then, the platelets unicycle around bumps along the healthy endothelium as
well. This also implies that RBCs indirectly assist in platelet activation for these
shear rates.

We also consider an alternative interpretation of the bumpy wall. Because the bumps are
approximately the same size as a platelet, we can consider this to be a rough model of
a subendothelium with a few deposited platelets. Under this interpretation, we view
platelet-wall contact as interactions between an unactivated platelet and the
subendothelium, when contact occurs in a valley, or between an unactivated platelet and
an between an unactivated and activated platelet adhering to the subendothelium, when
contact occurs on a bump.  These interactions correlate with a reduction in velocity at
the contact zone and the platelet membrane becomes flattened at the point of contact. We
suggest that the decreased velocity may be sufficient to allow bonds to form between the
platelets or between the platelet and subendothelium. By flattening, the platelet exposes
more surface area at the point of contact, so that the activation signals are more likely
to reach the platelet. The observed velocity reduction seems insufficient for this, but
the resolution of our simulations is also unlikely to allow cells to pass within bonding
distance of one another. Overcoming this limitation we leave as a future direction.
