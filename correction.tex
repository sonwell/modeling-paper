\section{Boundary error correction for staggered grids}\label{sec:boundary-correction}

The marker-and-cell (MAC) grid~\cite{Welch:1965jv} is a popular method for fluid
simulations. Components of vector-valued quantities are discretized at the center of the
corresponding cell faces and scalar-valued quantities at the cell center. This
staggering is a distinguishing feature of the MAC grid. Staggering avoids the
checkerboard instability that arises from using collocated grids~\cite{Wesseling:2001ci}. 
However, in domains with non-periodic boundaries, this means that some vector components
will encounter situations where satisfying boundary conditions with linear ghost value
extrapolation leads to numerical error. This appendix explores these errors and provides
a resolution that maintains compatibility with the conjugate gradients method.

\subsection{A simple case}

To illustrate the need for boundary correction, we consider a simplified problem. Because
every linear solve in the fluid solver depends on some form of the discrete Laplacian, we
focus our attention on that operator. Consider the 1-dimensional diffusion problem for
$u = u(x, t)$,
\begin{alignat}{2}
    u_t      &= \mu u_{xx} + f           &&\quad \text{for} \ x\in(0,\,1), \label{eq:1d-diff} \\
    \gamma_0 &= \alpha_0 u + \beta_0 u_x &&\quad \text{at} \ x=0, \label{eq:1d-0bcs} \\
    \gamma_1 &= \alpha_1 u - \beta_1 u_x &&\quad \text{at} \ x=1, \label{eq:1d-1bcs}
\end{alignat}
where $\alpha_m^2 + \beta_m^2 \neq 0$ for $m = 0,\,1$. Here, subscripts $t$ and $x$
indicate partial differentiation with respect to time and space, respectively. We
discretize $[0,\,1]$ into $N$ cells of length $h=1/N$ and let $u_i^n$ approximate
$u(x_i,\,n\timestep)$ at $x_i = h(g + i-1)$, where $i$ ranges over $\range{N}$, time $t =
n\timestep$, and $g\in(0,\,1]$ is the one-dimensional grid staggering. Consider the
boundary at $x=0$.  Let $x_0 = h(g-1)$ be a ghost point to the left of the boundary, and
let $u^n_0$ be the extrapolated value of $u$ at the ghost point. Near the boundary, we
use the linear approximations
\begin{equation}\label{eq:1d-approx}
    u(0, n\timestep) \approx au^n_1 + bu^n_g \quad \text{and} \quad u_x(0, nk) \approx a'u^n_1 + b'u^n_g.
\end{equation}
We expect $a'$ and $b'$ to be of order $\mathcal{O}(h^{-1})$, and require that these
approximations be at least first order:
\begin{equation}\label{eq:1st-order}
    a+b=1, \quad a'+b'=0,\ \quad \text{and} \quad a'hg + b'h(g-1) = 1.
\end{equation}
For simplicity, we drop the subscripts from $\alpha_0$, $\beta_0$, and $\gamma_0$, and
the superscripts indicating the timestep. Substituting into the boundary condition yields
\begin{equation}\label{eq:1d-bc}
    \gamma = \alpha(au_1 + bu_0) + \beta(a'u_1 + b'u_0) + \mathcal{O}(h).
\end{equation}
Given a value $u_1$ and boundary data $\gamma$, we can extrapolate
\begin{equation}
    u_0 \approx (\alpha b + \beta b')^{-1}(\gamma-(\alpha a + \beta a')u_1),
\end{equation}
when $\alpha b + \beta b' \neq 0$. In the extraordinary case that this does not hold, the
boundary condition is of Robin type, with neither $\alpha$ nor $\beta$ zero, and value at
the ghost point is arbitrary.  We then have four linearly independent conditions for the
weights $a$, $b$, $a'$, and $b'$: the three conditions in~\eqref{eq:1st-order} and
$\alpha b + \beta b' = 0$. Solving for the weights reduces~\eqref{eq:1d-bc} to
$\alpha u_1 = \gamma$. We do not consider this case further.

Assuming $\alpha b + \beta b' \neq 0$, the standard 3-point discrete Laplacian at $x_1$
gives the approximation
\begin{equation}
    \label{eq:1d-3ptl}
    u_0 - 2u_1 + u_2
    = (\alpha b + \beta b')^{-1}\gamma - \left(2 + (\alpha b + \beta b')^{-1}(\alpha a + \beta a')\right)u_1 + u_2.
\end{equation}
Replacing approximations with exact values and Taylor expanding about $x_1$ yields
\begin{equation}
    \label{eq:1d-3ptl-expanded}
    \begin{aligned}
        u_0 - 2u_1 + u_2
        &= (\alpha b + \beta b')^{-1}\left(\alpha\left(u-hgu_{xx} + \sfrac12(hg)^2 u_{xx}\right) + \beta(u_x - hgu_{xx})\right) \\
        &\quad -\left(2+(\alpha b+\beta b')^{-1}(\alpha a + \beta a')\right)u + \left(u + hu_x + \sfrac12 h^2u_{xx}\right) + \mathcal{O}(h^3) \\
        &= (\alpha b + \beta b')^{-1}(\alpha(1-a-b) - \beta(a'+b'))u \\
        &\quad +(\alpha b + \beta b')^{-1}(\beta - \alpha hg + (\alpha b + \beta b')h)u_x \\
        &\quad +(\alpha b + \beta b')^{-1}\left(\sfrac12\alpha(gh)^2-\beta hg +\sfrac12(\alpha b + \beta b')h^2\right)u_{xx} + \mathcal{O}(h^3).
    \end{aligned}
\end{equation}
The coefficient of $u$ vanishes according to~\eqref{eq:1st-order}. We further require
that the coefficient of $u'$ be zero. That is,
\begin{equation*}
    \alpha b + \beta b' = -h^{-1}(\beta-\alpha hg).
\end{equation*}
Choosing $a=1-g$, $b=g$, $a'=h^{-1}$, and $b'=-h^{-1}$ satisfies these conditions. As a
result,
\begin{equation*}
    \alpha a + \beta a' = h^{-1}(\beta + \alpha h(1-g)).
\end{equation*}
Finally, the first possibly nonzero coefficient is that of $u_{xx}$:
\begin{equation}
    \begin{aligned}
        h^{-2}\left[u_0 - 2u_1 + u_2\right]
        &\hphantom{:}= \left[\sfrac12 + g(\beta-\alpha hg)^{-1}(\beta-\sfrac12 hg)\right]u_{xx} + \mathcal{O}(h^2) \\
        &\hphantom{:}= \left[1-\sfrac12(\beta-\alpha hg)^{-1}(\beta(1-2g)-\alpha hg(1-g))\right]u_{xx} + \mathcal{O}(h^2) \\
        &:= (1-\epsilon)u_{xx} + \mathcal{O}(h^2).
    \end{aligned}
    \label{eq:lap-error}
\end{equation}
Near the boundary, when $\epsilon\neq0$, \latin{i.e.}, $\beta(1-2g)-\alpha hg(1-g)\neq0$,
these approximations yield a $0^\text{th}$-order approximation to the Laplacian. Cases
where $\epsilon \neq 0$ arise naturally from using staggered grids in a domain with at
least one non-periodic dimension. In fact, for fixed $\alpha$ and $\beta$, only
$g=g^\ast(\sfrac{2\beta}{\alpha h})$ results in $\epsilon = 0$, where
\begin{equation*}
    g^\ast(r) = \begin{cases}
        \sfrac12\left(1+r+\sqrt{1+r^2}\right), & r \le 0 \\
        \sfrac12\left(1+r-\sqrt{1+r^2}\right), & r > 0.
    \end{cases}
\end{equation*}
For Neumann boundaries, $\epsilon = 0$ when $g = 0.5$; for Dirichlet boundaries, when
$g = 1$. The case for the opposing boundary is very similar: simply substitute the
correct boundary condition coefficients and data, $-h$ for $h$, and when $g\neq1$, $1-g$
for $g$. We will use $\epsilon_0$ and $\epsilon_1$, when necessary, to distinguish the
error factor when approximating the Laplacian at the $x=0$ and $x=1$ boundaries,
respectively.

\begin{figure}[t]
%\tikzexternalenable
\centering
\begin{tikzpicture}
    \begin{axis}[view={0}{90}, colorbar horizontal, xmin=0, xmax=0.5, ymin=0.01, ymax=0.99, ylabel=$x$, xlabel=$t$, scale only axis, width=1.95in, clip=false]
        \addplot3[surf, mesh/cols=100, shader=interp] file {dirichlet.dat};
        \node at (0.45, 0.9) {(a)};
    \end{axis}
\end{tikzpicture}
\hspace{0.5cm}
\begin{tikzpicture}
    \begin{axis}[view={0}{90}, colorbar horizontal, xmin=0, xmax=0.5, ymin=0.01, ymax=0.99, xlabel=$t$, ytick=\empty, scale only axis, width=1.95in, clip=false]
        \addplot3[surf, mesh/cols=99, shader=interp] file {neumann.dat};
        \node at (0.45, 0.9) {(b)};
    \end{axis}
\end{tikzpicture}
\caption[Propagation of boundary errors without correction]{%
    Propagation of errors near the boundary in approximating the solution of
    $u_t = u_{xx} + 1$ on $[0,\,1]$ without correction at the boundary. Initially, $u$ is
    analytically steady: $u(x,\,0) = x(1-x)/2$. We expect no change in $u$ over time.
    (a) $u$ satisfies homogeneous Dirichlet boundary conditions. The domain is
    discretized using $h=0.01$, with points $x_i=h(i-0.5)$ for $i=1,\,\ldots,\,100$.
    (b) $u$ satisfies the Neumann boundary conditions $u_x(0,\,t)=-u_x(1,\,t)=\sfrac12$.
    The domain is discretized using $h=0.01$, with points $x_i = hi$ for
    $i=1,\,\ldots,\,99$.
}\label{fig:error}
%\tikzexternaldisable
\end{figure}

Suppose that in approximating the solution to~\eqref{eq:1d-diff}--\eqref{eq:1d-1bcs}, we
employ the Crank-Ni\-col\-son timestepping scheme. The discrete equations are
\begin{equation}
    \frac{u_i^{n+1}-u_i^n}{\timestep} = \frac{\mu}2\left(\frac{u_{i-1}^{n+1}-2u_i^{n+1}+u_{i+1}^{n+1}}{h^2} + \frac{u_{i-1}^n-2u_i^n+u_{i+1}^n}{h^2}\right) + f_i^{n+1/2},
    \label{eq:disc-1d-diff}
\end{equation}
where superscripts denote the time step and subscripts the space step. With
$\lambda=\mu\timestep$, we rewrite this in matrix form as
\begin{equation}\label{eq:disc-1d-diff-mat}
    (I-\sfrac12\lambda \laplacian_h)\vec{u}^{n+1} = (I+\sfrac12\lambda \laplacian_h)\vec{u}^n + \lambda B_h\vec{\gamma}^{n+1/2} + \timestep\vec{f}^{n+1/2},
\end{equation}
where $B_h$ modifies equations near the boundary with boundary data, according to~%
\eqref{eq:1d-3ptl}. As we have shown, this introduces an error near the boundary.
This error will propagate into the center of the domain at a rate dependent upon $\mu$.
Figure~\ref{fig:error} illustrates this phenomenon. The convergence test in Table~%
\ref{tab:dir-bdy-conv} shows that these errors vanish at second order for Dirichlet
boundary conditions. While Dirichlet boundary errors vanish at second order, the method
converges to the wrong steady state. It is clear from Figure~\ref{fig:error}(b) that
despite the first-order convergence shown in Table~\ref{tab:neu-bdy-conv}, Neumann
boundary errors will only grow as the simulation progresses.

\begin{table}[t]
\centering
\caption[Convergence of Dirichlet boundary errors on a staggered grid]{%
    Convergence test for Crank-Nicolson timestepping without boundary correction for the
    test problem $u_t = u_{xx} + 1$ with initial steady conditions $u(x,\,0) = x(1-x)/2$
    and homogeneous Dirichlet boundary conditions.
}\label{tab:dir-bdy-conv}
\begingroup
\setlength{\tabcolsep}{9pt}
\renewcommand{\arraystretch}{1.5}
\begin{tabular}{cc|cc|cc}
    \toprule
    $N$ & $\timestep$ (ms) & $\|u-u_0\|_2$ &   order & $\|u-u_0\|_{\infty}$ &    order \\ \midrule
     25 & 4                &   142.438496  &         &           194.026473 &          \\
     50 & 2                &    35.639706  & 1.99878 &            49.254058 &  1.977949 \\
    100 & 1                &     8.911804  & 1.99969 &            12.406781 &  1.989114 \\
    200 & 0.5              &     2.228068  & 1.99992 &             3.113348 &  1.994590 \\
    \bottomrule
\end{tabular}
\endgroup
\end{table}

\begin{table}[t]
\centering
\caption[Convergence of Neumann boundary errors on a staggered grid]{%
    Convergence test for Crank-Nicolson timestepping without boundary correction for the
    test problem $u_t = u_{xx} + 1$ with initial steady conditions $u(x,\,0) = x(1-x)/2$
    and Neumann boundary conditions $u_x(0,\,t) = -u_x(1,\,t) = \sfrac12$.
}\label{tab:neu-bdy-conv}
\begingroup
\setlength{\tabcolsep}{9pt}
\renewcommand{\arraystretch}{1.5}
\begin{tabular}{cc|cc|cc}
    \toprule
    $N$ & $\timestep$ (ms) & $\|u-u_0\|_2$ &   order & $\|u-u_0\|_{\infty}$ &   order  \\ \midrule
     25 & 4                &   4308.714364 &         &          6950.600999 &          \\
     50 & 2                &   2141.879730 & 1.00838 &          3559.181576 & 0.965592 \\
    100 & 1                &   1067.893238 & 1.00411 &          1801.330925 & 0.982482 \\
    200 & 0.5              &    533.192720 & 1.00204 &           906.192341 & 0.991174 \\
    \bottomrule
\end{tabular}
\endgroup
\end{table}


\subsection{One-dimensional correction}

To improve the approximations near the boundary, we replace the corresponding rows of~%
\eqref{eq:disc-1d-diff-mat} with those obtained by discretizing the scaled equation
\begin{equation}\label{eq:scaled}
    (1-\epsilon)u_t = \mu(1-\epsilon)u_{xx} + (1-\epsilon)f.
\end{equation}
The solution should be identical to that of the original equation as long as
$\epsilon\neq 1$, but the Laplacian constructed above need not be modified to approximate
$(1-\epsilon) u_{xx}$. We simply multiply the remaining terms by $1-\epsilon$. Define the
modified identity matrix
\begin{equation}\label{eq:mod-ident}
    \tilde{I} = \left[\begin{array}{ccccc}
            1-\epsilon_0 &   &        &   &         \\
                         & 1 &        &   &         \\
                         &   & \ddots &   &         \\
                         &   &        & 1 &         \\
                         &   &        &   & 1-\epsilon_1
            \end{array}\right].
\end{equation}
Rescaling equations for values near the boundary, equation~\eqref{eq:disc-1d-diff-mat}
becomes
\begin{equation}\label{eq:1d-diff-mat}
    (\tilde{I}-\sfrac12\lambda \laplacian_h)\vec{u}^{n+1} = (\tilde{I}+\sfrac12\lambda \laplacian_h)\vec{u}^n + \lambda B\vec{\gamma}^{n+1/2} + \timestep\tilde{I}\vec{f}^{n+1/2}.
\end{equation}
This improves the error near the boundary to second-order at the cost of one more
diagonal matrix-vector multiplication.

Alternatively, one could approximate the Laplacian near the boundary using a quadratic
interpolant. It would always give a second-order approximation but would break the
symmetry of the discrete Laplacian. Linear interpolation maintains symmetry, and the
coefficients obtained near the boundary are exactly those of the quadratic interpolant,
scaled by $1-\epsilon$. The correction recovers the solutions to the problems illustrated
in Figure~\ref{fig:error} to machine precision. By modifying only the identity matrix,
the discrete Helmholtz operator, $\tilde{I}-\sfrac12\lambda \laplacian_h$, maintains its
symmetry, and $\epsilon < 1$ is sufficient to maintain positive-definiteness. Many types
of boundary conditions satisfy $\epsilon < 1$, most notably all boundary conditions of
Dirichlet or Neumann type. We can therefore continue to use conjugate gradients for the
linear solves.

\subsection{Higher-dimensional correction}

We now consider a higher-dimensional Laplacian. Construction, and therefore correction,
proceeds recursively, by analog to the continuous Laplacian. For example, the
three-dimensional Laplacian is the sum of the two-dimensional Laplacian in $x$ and $y$
and the second-derivative operator with respect to $z$. The discrete analog of adding
operators is the tensor sum, e.g.,
\begin{equation*}
    L_y \oplus L_x = I_y \odot L_x + L_y \odot I_x,
\end{equation*}
where $L_x$ and $L_y$ are square, one-dimensional discrete second-derivative operators
with respect to $x$ and $y$, respectively; $I_x$ and $I_y$ are identity operators the
same size as $L_x$ and $L_y$, respectively; and $\odot$ is the Kronecker tensor product.
The Kronecker tensor product takes two square matrices, $A = (a_{ij}) \in \R^{n\times n}$
and $B \in \R^{m\times m}$, and produces the $mn\times mn$ block matrix
\begin{equation*}
    A \odot B = \left[\begin{array}{cccc}
            a_{11} B & a_{12} B & \hdots & a_{1n} B \\
            a_{21} B & a_{22} B & \hdots & a_{2n} B \\
            \vdots   & \vdots   & \ddots & \vdots   \\
            a_{n1} B & a_{n2} B & \hdots & a_{nn} B
    \end{array}\right].
\end{equation*}
If $A$ and $B$ are also symmetric, so is $A \odot B$. The two-dimensional discrete
identity operator is also constructed via tensor product: $I_y \odot I_x$. The
$d$-dimensional discrete Laplacian and identity operators are computed recursively via
\begin{equation}\label{eq:hi-d-operators}
    \begin{aligned}
        &L^{[d]} = I^{[1]} \odot L^{[d-1]} + L^{[1]} \odot I^{[d-1]} = L^{[1]} \oplus L^{[d-1]}, \\
        &I^{[d]} = I^{[1]} \odot I^{[d-1]},
    \end{aligned}
\end{equation}
where the superscript indicates the dimensionality of the operators.

Consider a two-dimensional diffusion problem on the domain $\domain = [0,\,1]^2$,
\begin{alignat}{2}
    u_t      &= \mu\left(\frac{\partial^2}{\partial x^2} + \frac{\partial^2}{\partial y^2}\right) u + f     &&\quad \text{in} \ \domain, \label{eq:2d-diff} \\
    \gamma   &= \mathcal{B}u             &&\quad \text{on} \ \partial\domain \label{eq:2d-bdy},
\end{alignat}
where $\mathcal{B}$ is a boundary operator.  We imagine the case where either one of the
second derivatives composing the Laplacian requires correction at the boundary. Without
loss of generality, we will assume they both do. We scale Equation~\eqref{eq:2d-diff} as
we did in Equation~\eqref{eq:scaled}:
\begin{equation}\label{eq:2d-scale-x}
    (1-\epsilon_x) u_t = \mu\left(\frac{\partial^2}{\partial x^2} + (1-\epsilon_x)\frac{\partial^2}{\partial y^2}\right) u + (1-\epsilon_x) f
\end{equation}
near an $x$ boundary,
\begin{equation}\label{eq:2d-scale-y}
    (1-\epsilon_y) u_t = \mu\left((1-\epsilon_y)\frac{\partial^2}{\partial x^2} + \frac{\partial^2}{\partial y^2}\right) u + (1-\epsilon_y) f
\end{equation}
near a $y$ boundary, and
\begin{equation}\label{eq:2d-scale-xy}
    (1-\epsilon_{xy}) u_t = \mu\left((1-\epsilon_y)\frac{\partial^2}{\partial x^2} + (1-\epsilon_x)\frac{\partial^2}{\partial y^2}\right) u + (1-\epsilon_{xy}) f
\end{equation}
near $x$ and $y$ boundaries, where $\epsilon_x$ and $\epsilon_y$ correspond to the error
in the discrete second $x$ and $y$ derivatives, respectively, and
$1-\epsilon_{xy} = (1-\epsilon_x)(1-\epsilon_y)$. Let the subscripts $x$ and $y$ denote
the 1-dimensional operator for the $x$ and $y$ dimension, respectively, of the modified
identity, $\tilde{I}$, second derivative, $L$, and boundary operator $B$. We write the
discretization of Equations~\eqref{eq:2d-scale-x}--\eqref{eq:2d-scale-xy} with
Crank-Nicolson timestepping succinctly as
\begin{multline}\label{eq:2d-corr-disc}
    (\tilde{I}_y \odot \tilde{I}_x) \frac{\u^{n+1} - \u^n}{\timestep} =
    \frac{\mu}{2}(\tilde{I}_y \odot L_x + L_y \odot \tilde{I}_x)(\u^{n+1} + \u^n) \\
    +(\tilde{I}_y \odot B_x + B_y \odot \tilde{I}_x)\gamma^{n+\sfrac12} + (\tilde{I}_y \odot \tilde{I}_x) f^{n+\sfrac12},
\end{multline}
Letting $\tilde{I} = \tilde{I}_y \odot \tilde{I}_x$,
$\laplacian_h = \tilde{I}_y \odot L_x + L_y \odot \tilde{I}_x$, and
$B_h = \tilde{I}_y \odot B_x + B_y \odot \tilde{I}_x$, we can express Equation~%
\eqref{eq:2d-corr-disc} identically to its one-dimensional analog, Equation~%
\eqref{eq:1d-diff-mat}. For boundaries that do not require correction,
$\tilde{I}\equiv I$, and the recursion~\eqref{eq:hi-d-operators} can be used to construct
even higher-dimensional operators by replacing identity operators with their
boundary-corrected counterparts. The system is symmetric positive-definite if every
$\epsilon > 0$, and can be solved via conjugate gradients.

% vim: cc=90 tw=89
