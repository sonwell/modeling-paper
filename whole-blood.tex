\subsection{Effect of endothelial shape on platelet motion}\label{sec:whole-blood}

Throughout this section, we consider a $16\um\times12\um\times16\um$ domain, which is
again periodic in the $x^1$ and $x^3$ dimensions and has Dirichlet boundary conditions on
the $x^2$ dimension. Along the bottom of the domain, the fluid velocity is zero, and
the top of the domain has a constant velocity such that, in the absence of structures,
the steady fluid velocity has a shear rate of $\dot{\gamma} = 1000\si{\per\second}$.

Eight RBCs are placed in the space above the endothelium, aligned in two staggered rows
in the $\e_3$ direction and then slightly randomly perturbed. Initially, we do not place
platelets in the domain, and begin with a flow in the $\e_3$ direction. Over $7.2\ms$,
the top boundary condition is rotated to drive flow in the  $(\e_1 + \e_2)/\sqrt{2}$
direction. Around $16$--$19\ms$, the first RBC overtakes its neighbor. In this time, the
RBCs have taken on a shape similar to those in Figure~\ref{fig:tread}.

\subsubsection{Platelets along a flat wall}
\subsubsection{Whole blood along a flat wall}
\subsubsection{Whole blood along elongated endothelial cells}
\subsubsection{Whole blood along cobblestone endothelial cells}
