\section{The immersed boundary method}\label{sec:ib}

\subsection{Overview}\label{sec:ib_old}

Consider a rectangular domain, $\domain\subset\R^3$, which contains one or more deformable structures and is
otherwise filled with an incompressible, Newtonian fluid with constant density, $\density$, and viscosity,
$\viscosity$. The IB method treats these structures as an extension of the fluid. The motion of any particle in
$\domain$ is therefore governed by the incompressible Navier-Stokes equations,
\begin{gather}
    \density\left(\frac{\partial\u}{\partial t} + \div(\u\otimes\u)\right) = \viscosity\laplacian \u - \grad p + \f, \label{eq:ins-momentum} \\
    \div \u = 0, \label{eq:ins-incomp}
\end{gather}
where, for $\x = (x, y, z) \in \domain$, $\u = \u(\x, t) = (u, v, w)$ is the fluid velocity, $p = p(\x, t)$ is the
pressure, and $\f = \f(\x, t)$ is an external force density. Here and throughout this paper, we use bold italic
symbols to indicate vectors in $\R^3$. Treating the entire domain as a fluid allows us to discretize $\domain$
independently of the immersed structures with a fixed Eulerian grid of spacing $h$. The discretization of~%
\eqref{eq:ins-momentum} and~\eqref{eq:ins-incomp} is discussed in detail in \cref{sec:ins}.

Let $\X = \X(\theta, \varphi, t)$, for surface coordinates $(\theta, \varphi)$ in $\sites\subset\R^2$, be a
parametrization for the immersed boundary $\interface$. In the IB method, a Lagrangian representation is used to
track immersed boundaries.  Nevertheless, in continuum form, $\X$ satisfies a no-slip condition and moves with the
background fluid. The velocity of $\interface$ can therefore be represented as a convolution of the local fluid
velocity against the Dirac delta function $\Dirac(\x)$, \latin{i.e.},
\begin{equation}\label{eq:ib-interp}
    \frac{\partial \X}{\partial t} = \U(\X, t) % \U because we need to introduce some nice notation for surface velocity for the energies section
                                               % If we need to change this, CHANGE THE MACRO.
        = \int\limits_{\domain} \u(\x, t) \Dirac(\x-\X) \d\x,
\end{equation}
where $\U$ is the restriction of $\u$ to $\interface$.
The process of transferring the fluid velocity to the immersed structure is termed ``interpolation\qend.'' % or
%``restriction\qend.''
As a boundary deforms, it generates a force density $\F = \F(\X, t)$, which it imparts onto
the fluid as $\f$ in~\eqref{eq:ins-momentum}. Similar to~\eqref{eq:ib-interp}, $\F$ is transferred to the fluid at
$\x$ via
\begin{equation}\label{eq:ib-spread}
        \f(\x, t)
        = \int\limits_{\interface} \F(\X, t)\Dirac(\x-\X) \d\X.
\end{equation}
This transfer of forces is called ``spreading\qend.'' The structure $\interface$ is typically represented in
Lagrangian form, usually as a discrete set of points. Consequently,~\eqref{eq:ib-interp} must be discretized to
update those points based on their velocities. Similarly,~\eqref{eq:ib-spread} must be discretized to both compute
$\F(\X, t)$ and to approximate the integral of $\F(\X, t)$ over $\interface$. Finally, the Dirac delta
$\Dirac(\x)$ is typically replaced by a smoothed, compactly-supported, $h$-dependent analog, typically referred to
as the ``discrete Delta function'' $\Dirac_h(\x)$.

In the classical version of the IB method~\cite{Peskin:2002go}, velocities are interpolated to the same Lagrangian
points that forces are spread from.  However, there exist several IB methods which instead use different sets of
points for each operation instead. For instance, Griffith \& Luo~\cite{Griffith:2017id} use a finite element
representation for the structure, and consequently spread forces from (Lagrangian) quadrature points, interpolate
velocities to quadrature points, and then project these using the finite element basis to the individual element
nodes (which are also Lagrangian points). The RBF-IB method~\cite{Shankar:2015km} used in this work does something
similar. In \cref{sec:rbfib}, we discuss the discretization of~\eqref{eq:ib-interp} and~\eqref{eq:ib-spread} in
the context of the RBF-IB method.

\subsection{The RBF-IB method}\label{sec:rbfib}

The RBF-IB method obtains its name from the fact that it uses meshless interpolation with RBFs to parametrize the
structure $\interface$ as $\X(\theta, \varphi, t)$. However, it also distinguishes itself from the classical IB
method by using two sets of Lagrangian points to represent the structure. One set of points, which we term
``movement points'' and label $\data\X_k$, $k=1, \ldots, \data\cardinality$, are used to move the structure in the
discrete analog to~\eqref{eq:ib-interp}. Thus, the fully discrete version of~\eqref{eq:ib-interp} in the RBF-IB
method reads
\begin{equation}\label{eq:ib-interp-disc}
    \frac{\partial\data\X_k}{\partial t} = \U(\data\X_k, t) \approx h^3 \sum_i \u(\x_i, t) \Dirac_h(\x_i-\data\X_k), \quad k=1, \ldots, \data\cardinality,
\end{equation}
where $h$ is the spacing of the background Eulerian grid and $i$ enumerates Eulerian grid points. %, and the $i$ Eulerian grid points are implicitly chosen
%by the support of the discrete Delta function $\Dirac_h$ centered at $\data\X_k$.
In this work, we select the
movement points to be initially approximately $2h$ apart in $\R^3$ at the start of a simulation. The RBF-IB method
also uses a second set of points, which we term ``spreading points'' and label $\Xsj$,
$j=1, \ldots, \sample\cardinality$. These points are used to discretize~\eqref{eq:ib-spread} in order to spread
forces to the background Eulerian grid. The fully discrete version of~\eqref{eq:ib-spread} in the RBF-IB method
for an Eulerian grid point $\x_i$ can therefore be written as
\begin{align}\label{eq:ib-spread-disc}
        \f(\x_i, t) \approx\sum\limits_{j=1}^{\sample\cardinality} \weight[j]\F(\Xsj) \Dirac_h(\x_i-\Xsj),
\end{align}
where $\weight[j]$, $j=1, \ldots, \sample\cardinality$ is a set of quadrature weights for integrating quantities
at the spreading points $\sample\X$; the spreading points can therefore be thought of as a set of quadrature
points.% As in~\eqref{eq:ib-interp-disc}, the set of Eulerian points $\x$ affected by a spreading point $\Xsj$ in~%
%\eqref{eq:ib-spread-disc} is determined by the support of $\Dirac_h$.

In general, the spreading points are selected to be approximately $h$ apart in $\R^3$ at the start of a
simulation; this spacing corresponds to spacing used for the single set of Lagrangian points (IB points) used in
traditional IB simulations. Thus, in the RBF-IB method, $\sample\cardinality > \data\cardinality$. In this sense,
the RBF-IB method is a generalization of the standard IB method. The use of movement points and spreading points
partially decouples the error in representing the structure $\interface$ from the error in integrating forces. The
spacing of the movement points $\data\X$ is chosen so that the reconstruction of $\interface$ from $\data\X$ and
the calculation of $\F(\X, t)$ by the RBF geometric model is sufficiently accurate. On the other hand, the spacing
of the spreading points $\sample\X$ is chosen so that the numerical integration of $\F(\X, t)$ is sufficiently
accurate, to prevent leaks, and to ensure that spread forces overlap sufficiently on the Eulerian grid. In this
work, we identify a set of fixed parametric locations $\data\sites = \{(\theta_k, \varphi_k)\}$,
$k=1, \ldots, \data\cardinality$ with the movement points $\data\X = \X(\data\sites, t)$, and call them ``data sites\qend.''
Analogously, we also maintain a set of fixed parametric points $\sample\sites = \{(\theta_j, \varphi_j)\}$,
$j=1, \ldots, \sample\cardinality$, that we identify with the spreading points $\sample\X$; these points are
called ``sample sites\qend.'' In general, we need only explicitly track the movement points $\data\X$, reconstruct
$\interface$ from $\data\X$ and the $\data\sites$ using our RBF geometric model in order to compute $\F(\X, t)$,
and regenerate the spreading points $\sample\X$ at the end of every step using the $\sample\sites$ and the RBF
geometric model.

Despite the advantages mentioned above, the choice of two sets of Lagrangian points potentially raises certain
concerns, which we address here. The first major concern relates to the force-spreading and velocity interpolation
operations. In traditional IB methods, the corresponding spreading and interpolation operators are formally
adjoint, leading to conservation of energy/power~\cite{Peskin:2002go}. However, in the RBF-IB method, these
operators are no longer adjoint, leading to concerns about numerical stability. Nevertheless, in previous
work~\cite{Shankar:2015km}, the authors demonstrated that the RBF-IB method dissipated energy in 2D simulations as
expected, despite the absence of formal adjointness. In this work, we show similar results for an RBC relaxation
problem in a full 3D simulation in \cref{sec:energy-est}. The second major concern relates to the wider $2h$
spacing of the movement points $\data\X$. If the points are spaced too widely apart, it is possible to obtain
unphysical behavior, with some parts of the structure responding more to the neighboring fluid than others.
However, both in~\cite{Shankar:2015km} and in this work, we observed that starting these points with a spacing of
$2h$ was sufficient to reproduce standard results for RBC simulations. It may be possible to alleviate any spacing
issues by periodically rearranging the movement points $\data\X$, but this approach was not necessary in this
work, and so we leave a full exploration of such Lagrangian rearrangement strategies for future work.

On a higher level,~\eqref{eq:ib-spread-disc} illustrates the need for three pieces of information for each
immersed structure: the points $\Xsj$ used to evaluate forces on the structure; a force density $\F(\Xsj)$ at each
of those points; and surface integration or quadrature weights $\weight[j]$ for those points. We detail the
RBF-based meshless geometric model used to obtain $\Xsj$, $\F(\Xsj)$, and $\weight[j]$ in \cref{sec:rbfs}, give
analytic expressions for energy densities used to derive $\F$ in \cref{sec:energy}, and give analytic expressions
for the force densities $\F$ in~\ref{sec:forces}.
