%\begin{table*}[tbh]
%    \centering
%    \setlength{\tabcolsep}{9pt}
%    \renewcommand{\arraystretch}{1.5}
%    \begin{tabular}{l|c|l}
%                                                                                                               \toprule
%        \textbf{Description}                  & \textbf{Example} & \textbf{Meaning}                         \\ \midrule
%        Italic boldface                       & $\x$             & Cartesian vector                         \\
%        Super- and subscripts:                &                  &                                          \\
%        \hspace{0.4cm} Italic lowercase Latin & $u^a$            & Eulerian coordinate index                \\
%        \hspace{0.4cm} Italic lowercase Greek & $\theta^\alpha$  & Lagrangian coordinate index              \\
%        \hspace{0.4cm} Italic uppercase Latin & $\X_A$           & Set enumeration index                    \\
%        \hspace{0.4cm} Upright Latin          & $\sample\sites$  & Differentiates between sets or functions \\
%        Subscript comma                       & $\phi_{,\alpha}$ & Derivative in a coordinate direction     \\
%        Superposed dot                        & $\dot{\X}$       & Time derivative                          \\
%        Superposed hat                        & $\hat{H}$        & Reference value                          \\ \bottomrule
%    \end{tabular}
%    \caption{%
%        A summary of the notation used throughout this paper.
%    }
%    \label{tab:notation}
%\end{table*}

\section{The immersed boundary method}\label{sec:ib}

Consider a rectangular domain, $\domain\subset\mathbb{R}^3$, which contains one or more
deformable structures and is otherwise filled with an incompressible, viscous fluid with
constant density, $\density$, and viscosity, $\viscosity$. The IB method treats these
structures as an extension of the fluid. The motion of any particle in $\domain$ is
therefore governed by the incompressible Navier-Stokes equations,
\begin{gather}
    \density(\u + \div(\u\otimes\u)) = \viscosity\laplacian \u - \grad p + \f, \label{eq:ins-momentum} \\
    \div \u = 0, \label{eq:ins-incomp}
\end{gather}
where, for $\x = (x,\,y,\,z) \in \domain$, $\u = \u(\x,\,t) = (u, v, w)$ is the fluid
velocity, $p = p(\x,\,t)$ is the pressure, and $\f = \f(\x,\,t)$ is an external force
density. Treating the entire domain as a fluid allows us to discretize $\domain$
independently of these structures--- the immersed boundaries---with a fixed regular grid
of spacing $h$. This is
the Eulerian grid.


Let $\X = \X(\qs,\,t)$, for surface coordinates $\qs$ in $\sites\subset\mathbb{R}^2$, be
a parametrization for the immersed boundary $\interface$. The boundary is impermeable to
the fluid and moves at the local fluid velocity. A discrete representation of the
boundary, consisting of a set of surface points, replaces its continuous counterpart.
Surface points are typically chosen to be within approximately $h$ of their neighbors.
This is the Lagrangian grid. Analytically convolving the fluid velocity against the Dirac
delta function, $\Dirac(\x)$, yields the velocity of a surface point, but discretely,
Eulerian and Lagrangian grid points are unlikely to coincide. The IB method replaces the
singular Dirac delta function with a smoothed, $h$-dependent analogue, $\Dirac_h(\x)$.
The Lagrangian point $\X$ evolves according to
\begin{equation}\label{eq:ib-interp}
    \vec{\dot{X}}
        = \int\limits_{\domain} \u(\x) \Dirac(\x-\X) \d\x
        \approx h^3 \sum_{i} \u(\x_i) \Dirac_h(\x_i-\X),
\end{equation}
where $i$ enumerates the Eulerian grid points, and a superposed dot denotes partial
differentiation with respect to $t$. As a boundary deforms, it generates a force density
$\F = \F(\qs,\,t)$, which it imparts onto the fluid as $\f$ in~ \eqref{eq:ins-momentum}.
By similar reasoning as the velocity, the $\F$ is transferred to the fluid at $\x$ via
\begin{equation}\label{eq:ib-spread}
        \f(\x)
        = \int\limits_{\interface} \F(\X)\Dirac(\x-\X) \d\X
        \approx \sum_{j} \weight[j]\F(\X_j) \Dirac_h(\x-\X_j),
\end{equation}
where $j$ enumerates the Lagrangian grid points, and $\weight[j]$ is the integration
weight corresponding to Lagrangian grid point $\X_j$.

In addition to a velocity vector field, equation~\eqref{eq:ib-spread} illustrates the
need for three pieces of information for each immersed structure: the position of points
on the structure, $\X_j$; a force density at each of those points, $\F(\X_j)$; and
surface integration weights, $\weight[j]$, for those points. Equation~\eqref{eq:ib-interp}
tells us how to update $\X_j$. The next three sections describe our methods for solving~%
\eqref{eq:ins-momentum}--\eqref{eq:ins-incomp}, give analytic expressions for $\F$, and
detail the discretization of the structure(s) to obtain $\F(\X_j)$ and $\weight[j]$,
respectively.
