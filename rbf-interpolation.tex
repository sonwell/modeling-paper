\subsection{Surface reconstruction with radial basis functions}\label{sec:rbf-interpolation}

We now describe our RBF method for reconstructing cell surfaces. RBF interpolation is a meshfree approach to
scattered data interpolation where structural information is encoded purely as point-wise distances. This
contrasts with, \latin{e.g.}, polynomials, where points must be chosen at grid vertices, or spherical harmonics,
for which special node sets are typically used. RBFs have been shown to be a viable approach for representing
cells on a par with Fourier methods~\cite{Shankar:2013ki}. They are therefore appealing for representing blood
cells and platelets.

Our goal is to produce a parametric reconstruction of cells. We choose to parametrize both red blood cells and
platelets on the 2-sphere, $\sphere$. We use a spherical coordinate mapping to relate Cartesian points on the
2-sphere to their corresponding parametric points. This mapping is given by
\begin{equation}
    \Xp(\theta, \varphi) =
    \begin{bmatrix}
        \cos\theta\cos\varphi \\
        \sin\theta\cos\varphi \\
        \sin\varphi
    \end{bmatrix},
\end{equation}
$(\theta, \varphi)\in(-\pi, \pi]\times[-\pi/2, \pi/2]$. As mentioned in Section~%
\ref{sec:rbfib}, let $\data\sites = \{(\theta_k, \varphi_k)\}$,
$k=1, \ldots, \data\cardinality$, be a set of distinct \emph{data sites}, defined by the
Bauer spiral~\cite{Bauer:2000km},
\begin{equation}\label{eq:bauer-spiral}
    \begin{aligned}
        &\varphi_k = \sin^{-1}(-1 + (2k - 1) / \data\cardinality),\\
        &\theta_k = \modulo\left(\sqrt{\data\cardinality\pi}\varphi_k + \pi, 2\pi\right) - \pi,
    \end{aligned}
\end{equation}
where $\modulo(a, b) = a - b\floor{a/b}$ is the modulo function. We similarly generate a larger set of
\emph{sample sites} from which to spread forces by replacing $\data\cardinality$ with $\sample\cardinality$ in the
above equation. Let $\data\Xp_i=\Xp(\theta_i, \varphi_i)$ for each $(\theta_i, \varphi_i)\in\data\sites$. In this
setting, our goal is to construct a parametric mapping from $\data\sites$ to the Cartesian locations of points on
cell surfaces. For RBC and platelet parametrizations, we identify the point $\X(\theta, \varphi, t)$ on
$\interface$ with the point $\Xp(\theta, \varphi)$ on $\sphere$. Each component of $\X$ and consequently each
component of a movement point $\data\X_k$ is then a function defined on $\sphere$. The problem of surface
reconstruction therefore involves approximating each of these functions from the $\data\sites$ using an RBF
interpolant. For the discussion that follows, we use $\psi(\Xp):\sphere\to\R$ to denote a function that we wish to
approximate.

Let $\phi:\sphere\times\sphere\to\R$ be a \emph{radial kernel} with the property that
$\phi(\Xp_i, \Xp_j)\equiv\phi(\|\Xp_i-\Xp_j\|)$. These kernels are sometimes called \emph{spherical basis
functions}. In addition to the radial kernels, let $p_k(\Xp)$, $k=1, \ldots, \poly\cardinality$ denote the first
$\poly\cardinality$ spherical harmonics, which form a natural basis for polynomial approximation on $\sphere$.
Then, the RBF interpolant to $\psi(\Xp)$ takes the form
\begin{equation}\label{eq:rbf-interp}
    s(\Xp)
    = \sum_{k=1}^{\data\cardinality} c_k \phi(\|\Xp-\data\Xp_k\|)
    + \sum_{i=1}^{\poly\cardinality} d_i p_i(\Xp),
\end{equation}
where $c_k$ and $d_i$ are unknown interpolation coefficients, and the metric can be written parametrically as
\begin{equation} \label{eq:param-metric}
    \begin{aligned}
    \|\Xp-\data\Xp_k\|
        &= \|\Xp(\theta, \varphi) - \Xp(\theta_k, \varphi_k)\| \\
        &= \sqrt{2-2\Xp(\theta, \varphi)\cdot\Xp(\theta_k, \varphi_k)} \\ % We can change the formatting, but we need this for the quadrature section
        &= \sqrt{2(1 - \cos\varphi\cos\varphi_k\cos(\theta-\theta_k) - \sin\varphi\sin\varphi_k)}.
    \end{aligned}
\end{equation}
To find $c_k$ and $d_i$, we enforce that the interpolant~\eqref{eq:rbf-interp} exactly interpolate the function
$\psi(\Xp)$ at each $\data\Xp_k$,
\begin{equation}
    s(\data\Xp_k) = \psi(\data\Xp_k), \qquad k=1, \ldots, \data\cardinality,\label{eq:interp_constraint}
\end{equation}
and that~\eqref{eq:rbf-interp} exactly reproduce the first $\poly\cardinality$ spherical harmonics everywhere on
$\sphere$~\cite{Fasshauer:2007ui},
\begin{equation}
    \sum_{k=1}^{\data\cardinality} c_k p_i(\data\Xp_k) = 0, \qquad i=1, \ldots, \poly\cardinality.\label{eq:constraints}
\end{equation}
We collect~\eqref{eq:interp_constraint} and~\eqref{eq:constraints} into a dense symmetric block system of the form
\begin{equation}\label{eq:rbf-interp-matrix}
    \begin{bmatrix}
            \Phi & P \\ P^T & 0
    \end{bmatrix}\begin{bmatrix}
            \arr{c} \\ \arr{d}
    \end{bmatrix} = \begin{bmatrix}
            \arr{\psi} \\ \arr{0}
    \end{bmatrix},
\end{equation}
where $\arr{c}$ and $\arr{d}$ are the unknown coefficients, $\Phi$ represents the evaluations of $\phi$, $P$
represents evaluations of the polynomials $p_k$, the matrix block $0$ is the
$\poly\cardinality\times\poly\cardinality$ zero matrix, and $\arr{0}$ is a vector of $\poly\cardinality$ zeros.
Because $\data\sites$ is fixed, we construct this matrix and compute its factors only once even though $\psi(\Xp)$
changes. The matrix in~\eqref{eq:rbf-interp-matrix} is invertible for any conditionally positive definite kernel
$\phi$ of order $m$ as long as the data sites $\data\sites$ are unisolvent for the first
$\poly\cardinality \ge (m+1)^2$ spherical harmonics, \latin{i.e.}, $P$ is of full rank~ \cite{Fasshauer:2007ui}. A
common heuristic choice to ensure this is to set $\data\cardinality = 2 \poly\cardinality$~\cite{SWJCP2018}.

It now remains to discuss the choice of $\phi$. In previous work~\cite{Shankar:2015km}, we chose $\phi$ to be an
infinitely-smooth and positive-definite kernel. While these kernels offer spectral convergence rates, they require
tuning a so-called shape parameter~\cite{Fasshauer:2007ui}. In particular, we instead choose $\phi$ to be a
polyharmonic spline (PHS), which is a conditionally-positive definite kernel with finite smoothness. In this work,
we set $\phi(r) = r^7$, augmented with fifth-order spherical harmonics to represent RBCs, or just the constant
spherical harmonic for platelets. As mentioned previously, we identify the point $\X(\theta, \varphi, t)$ on
$\interface$ with the point $\Xp(\theta, \varphi)$ on $\sphere$.  To reconstruct the surface $\X$ from the
movement points $\data\X$, we simply use the fact that each component of a particular movement point $\data\X_k$
is one of $\data\cardinality$ samples of a function of the form $\psi(\Xp)$. For instance, to reconstruct the
first component of $\X$, we may formally replace $\arr{\psi}$ with the vector of $x$-coordinates of the movement
points $\data\X$, and find a set of coefficients corresponding to the $x$-component of $\X$.  This process is
repeated component-wise for the movement points, and is essentially a component-wise interpolation of the movement
points~\cite{Shankar:2015km}.

Clearly, by replacing the quantity $\arr{\psi}$ with samples of any function defined on the sphere, one can also
use~\eqref{eq:rbf-interp} to interpolate other quantities on the surface $\interface$, since all such quantities
are functions of the form $\psi(\Xp)$. In this work, we also use~\eqref{eq:rbf-interp} to compute force densities
required by the IB method. It is clear from~\eqref{eq:skalak-law}--\eqref{eq:dissip-energy} that computing the
force densities requires values of $I_1$, $I_2$, and $H$, among others.  These values are derived from the first
and second derivatives of $\X$, which can be obtained by analytically differentiating the RBF and spherical
harmonic bases.

In practice, we do not compute any coefficients in our simulations. For efficiency, it is possible to reformulate
the process of interpolation followed by differentiation as a single application of a discrete (nodal)
differential operator (or a differentiation matrix). We discuss this in the next section.

\subsection{Discrete linear surface operators}

Let $\L$ be a linear operator. In particular, we are interested in the first- and second-order partial
differential operators, $\partial/\partial\theta$, $\partial^2/\partial\theta\partial\varphi$, \latin{etc}. First,
note that since $s(\Xp) \approx \psi(\Xp)$, it is also true that $\L s(\Xp) \approx \L \psi(\Xp)$. Thus, in the
RBF-IB method, it is of interest to be able to efficient apply $\L$ to $s(\Xp)$ and evaluate the resulting
function efficiently. To do so, we formulate all applications and evaluations of $\L s$ in terms of a
matrix-vector multiplication with the quantity $\arr{\psi}$. To see how this is done, first note that if we wish
to approximate samples of $\L s$ at a given set of sample sites $\sample\sites$, we can write
\begin{align}
\left.\L s (\Xp) \right|_{\sample\sites}
    = \sum_{k=1}^{\data\cardinality} c_k \L \phi(\|\Xp-\data\Xp_k\|)
    + \sum_{i=1}^{\poly\cardinality} d_i \L p_i(\Xp),
\end{align}
which can be expressed more compactly in terms of matrix-vector multiplications as
\begin{align}
\left.\L s (\Xp) \right|_{\sample\sites} =
    \begin{bmatrix}
    \L \Phi & \L P
    \end{bmatrix}\begin{bmatrix}
    \arr{c} \\
    \arr{d}
    \end{bmatrix}.
\end{align}
However, we can use~\eqref{eq:rbf-interp-matrix} to rewrite this as
\begin{align}
\left.\L s (\Xp) \right|_{\sample\sites} =
    \begin{bmatrix}
    \L \Phi & \L P
    \end{bmatrix}\begin{bmatrix}
    \Phi & P \\
    P^T & 0
    \end{bmatrix}^{-1}\begin{bmatrix}
    \arr{\psi} \\
    \arr{0}
    \end{bmatrix} =
    \begin{bmatrix}
    L & \ast
    \end{bmatrix}\begin{bmatrix}
    \arr{\psi}\\
    \arr{0}
    \end{bmatrix},
\end{align}
where $\L\Phi$ and $\L P$ represent evaluations of $\L\phi$ and $\L p_k$, respectively, and the
$\sample\cardinality \times \data\cardinality$ (dense) matrix $L$ is the discrete analog of $\L$. It is important
to note that $L$ is completely independent of the function $\psi$; in fact, it depends only on the functions
$\phi$ and $p_k$, the fixed data sites $\data\sites$, and the fixed sample sites $\sample\sites$. The block marked
by $\ast$ is multiplied by zeros, and can be discarded. We compute a separate $L$ for each operator $\L$ as a
preprocessing step, and simply apply these matrices to any function that needs to be evaluated or differentiated.
For instance, setting $\L$ to be the point evaluation operator and setting $\arr{\psi}$ to each component of the
movement points $\data\X$ allows us to generate each component of the spreading points $\sample\X$.  Similarly, we
can obtain samples of parametric derivatives of any function $\psi(\Xp)$ at $\sample\sites$ by replacing $\L$ with
that derivative operator.

It is also straightforward to generate versions of $L$ that produce derivatives at $\data\sites$ simply by
replacing $\left.\L s\right|_{\sample\sites}$ with $\left.\L s\right|_{\data\sites}$ in the above discussion. With
these operators in hand, the quantities $I_1$, $I_2$, and $H$ in Section~\ref{sec:energy} are readily discretized,
as are their corresponding force densities in~\ref{sec:forces}. Application of the dense discrete differential
operators is performed in parallel with a parallel implementation of \texttt{BLAS}. Lagrangian forces can
therefore be computed in parallel with few thread synchronizations.

We now have a method for computing a suitable set of points and for discretizing $\F$ for use in~%
\eqref{eq:ib-spread}. To compute a force from a force density, we need to compute the set of quadrature weights
$\weight[j]$, $j=1,\ldots,\sample\cardinality$. The following section is devoted to describing our method for
computing these weights.
