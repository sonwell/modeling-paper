\subsection{Surface reconstruction with radial basis functions}

RBFs have been shown to be a viable approach for representing cells on par with Fourier
methods~\cite{Shankar:2013ki}. Whereas methods such as finite element require a mesh or
other structured grid, RBFs are meshfree, and structural information is encoded purely as
pointwise distances. In order to compute the forces of the previous section, we must be
able to evaluate the interpolant and its first and second derivatives. We briefly
describe these methods below.

Consider the 2-sphere, $\sphere$, with parametrization
\begin{equation*}
    \Xp(\qs) = \cos\q[1]\cos\q[2]\e_1 + \sin\q[1]\cos\q[2]\e_2 + \sin\q[2]\e_3,
\end{equation*}
$\qs\in\sites = (-\pi,\,\pi]\times[-\pi/2,\,\pi/2]$. Let $\data\sites = \{\qs_i\}$ be a
set of $\data\cardinality$ distinct \emph{data sites}, and $\data\Xp_i = \Xp(\qs_i)$ for
each $\qs_i \in \data\sites$. We use the Bauer spiral~\cite{Bauer:2000km} to generate a
quasi-uniform set of $N$ points on the sphere in parallel:
\begin{equation}
    \begin{aligned}
        \q[2]_i &= \sin^{-1}(-1 + (2 i - 1) / N), \\
        \q[1]_i &= \pi - \modulo\left(\pi - \sqrt{\pi N} \q[2]_i,\,2\pi\right).
    \end{aligned}
    \label{eq:bauer-spiral}
\end{equation}
Suppose we wish to approximate $\psi(\Xp)$, a function defined on $\sphere$. With
\emph{basic function} $\phi$, the RBFs $\phi(\|\Xp-\data\Xp_A\|)$ form our basis.
Attractive choices for $\phi$ are the polyharmonic splines (PHS),
\begin{equation*}
    \text{PHS:}\quad\phi(r) = \begin{cases}
        r^k \log r, & \text{for even exponents } k \ge 2 \\
        r^k, & \text{for odd exponents } k \ge 1,
    \end{cases}
\end{equation*}
which do not require a shape parameter, unlike Gaussian or multiquadric kernels~%
\cite{Fasshauer:2007ui}. However, PHS are finitely differentiable and conditionally
positive definite; they require additional polynomial terms to guarantee a unique
interpolant. On $\sphere$, these are typically spherical harmonics. We denote these
$\poly\cardinality$ appended polynomials by $\{p_k(\Xp)\}$. To achieve a twice
continuously differentiable surface reconstruction, we are limited to PHS of order 5 or
higher. An interpolant $s(\Xp)$ exactly recovers data at each of the data sites and
therefore satisfies
\begin{equation}
    s(\data\Xp_j)
    := \sum_{i=1}^{\data\cardinality} c_i \phi\left(\left\|\data\Xp_j-\data\Xp_i\right\|\right)
    + \sum_{k=1}^{\poly\cardinality} d_k p_k(\data\Xp_j)
    = \psi(\data\Xp_j).
    \label{eq:interpolant}
\end{equation}
We further constrain $c_i$ so that the polynomials recover polynomial data,
\begin{equation}
    \sum_{i=1}^{\data\cardinality} c_i p_k(\data\Xp_i) = 0.
    \label{eq:constraints}
\end{equation}
These equations are gathered into a symmetric block system.

Suppose that $\X(\qs)$ is a parametrization of $\interface$, e.g., the membrane of an
RBC. We identify the point $\X(\qs)$ on $\interface$ with the point $\Xp(\qs)$ on
$\sphere$. Each component of $\X$ is therefore a function defined on $\sphere$.
By sampling $\X$ at each point in $\data\sites$, we can approximately reconstruct the
surface by interpolating each of the components. With a twice-differentiable interpolant
for the reference and deformed configurations of the surface, we can evaluate the force
densities of the previous section. For this, we require discrete differential operators.

Let $\L$ be a linear operator. In particular, we are interested in the differential
operators $\partial/\partial\q[\mu]$ and $\partial^2/\partial\q[\mu]\partial\q[\lambda]$.
We approximate $\L\psi$ by applying $\L$ analytically to $s$. Evaluating $\L s$ at each
data site involves dense $\data\cardinality\times(\data\cardinality+\poly\cardinality)$-%
matrix operations.
%Collecting values of
%I$\L\phi(\|\Xp-\data\Xp_i\|)$ and $\L p_k(\Xp)$ at each data site into a matrix results in
%a dense matrix multiplication to evaluate $\L s$.
To maintain a point spacing that avoids
leaking in the IB method, $\data\cardinality$ may be large. In the interest of saving
memory and time for such cases, we opt instead to use fewer data sites to reconstruct the
surface, and choose a larger set of $\sample\cardinality$ \emph{sample sites},
$\sample\sites$. We must then also consider $\L$ the identity operator.  Evaluating
$\L s$ at each sample site, we obtain
\begin{equation}
    \sum_{i=1}^{\data\cardinality} c_i \L\phi\left(\left\|\sample\Xp_j-\data\Xp_i\right\|\right) +
    \sum_{k=1}^{\poly\cardinality} d_k \L p_k(\sample\Xp_j) := (L \data{\vec{\psi}})_j
\end{equation}
where $\sample\Xp_j = \Xp(\qs_j)$, for each $\qs_j\in\sample\sites$, and
$\data{\vec{\psi}}$ is the vector of values obtained by evaluating $\psi$ at each data
site. The matrix $L$ is the discrete analogue of $\L$ applied at each sample site. We
perform this procedure for each first and second derivative operator, and for the
identity operator, to compute the points from which we spread the computed forces. In
other words, the points denoted $\X$ in equation~\eqref{eq:ib-interp} correspond to the
data sites, are tracked over the course of a simulations, and may be distinct from those
in equation~\eqref{eq:ib-spread}, which are derived from the sample sites. For fixed
$\data\sites$ and $\sample\sites$, we construct these operators only once.

The quantities $\metric_{\alpha\beta}$, $\metric^{\alpha\beta}$,
$\dot{\metric}_{\alpha\beta}$, $\n$, $H$, and $K$ require only local first and second
derivative data to compute. Ultimately, this means that after obtaining the necessary
geometric data, calculating forces according to~\eqref{eq:expanded-op} is trivially
parallelizable. The matrix $L$ is dense; its application can be performed in parallel
with, e.g., a parallelized implementation of BLAS, and so forces are computed in parallel
with few thread synchronizations.

We now have a method for discretizing $\F$ and for computing an appropriate set of points
for use in~\eqref{eq:ib-spread}. The following section is devoted to computing
integration weights, $\weight[i]$, at each sample site, $\X$, using RBFs.
