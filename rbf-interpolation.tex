\subsection{Surface reconstruction with radial basis functions}\label{sec:rbf-interpolation}

Radial basis functions (RBFs) are a meshfree approach to scattered data approximation
where structural information is encoded purely as point-wise distances. With some
exceptions, they are an appropriate tool for interpolation at arbitrary locations. This
contrasts with, \latin{e.g.}, polynomials, where points must be chosen at grid vertices,
or spherical harmonics, for which special node sets are typically used. RBFs are a viable
approach for representing cells on par with Fourier methods~\cite{Shankar:2013ki}. They
are therefore appealing for representing blood cells.

Because our choice of force model for the endothelium does not require geometric
information, we limit the discussion to RBCs and platelets, which are topologically
spherical. The 2-sphere, $\sphere$, has parametrization
\begin{equation*}
    \Xp(\theta,\,\varphi) =
    \left[\begin{array}{c}
        \cos\theta\cos\varphi \\
        \sin\theta\cos\varphi \\
        \sin\varphi
    \end{array}\right],
\end{equation*}
$(\theta,\,\varphi)\in(-\pi,\,\pi]\times[-\pi/2,\,\pi/2]$.
Let $\data\sites = \{(\data\theta_i,\,\data\varphi_i)\}$ be a set of $\data\cardinality$
distinct \emph{data sites}, defined by the Bauer spiral~\cite{Bauer:2000km},
\begin{equation}\label{eq:bauer-spiral}
    \begin{aligned}
        &\varphi_j = \sin^{-1}(-1 + (2i + 1) / N), \\
        &\theta_j = \modulo\left(\sqrt{N\pi}\varphi_j + \pi,\,2\pi\right) - \pi,
    \end{aligned}
\end{equation}
where $N$ is the number of points, and $\modulo(a,\,b) = a - b\floor{a/b}$ is the modulo
function. Let $\data\Xp_i = \Xp(\theta_i,\,\varphi_i)$ for each
$(\theta_i,\,\varphi_i) \in \data\sites$. Suppose we wish to approximate $\psi(\Xp)$,
defined on $\sphere$. From \emph{basic function} $\phi$, we form our interpolatory basis
with the RBFs $\phi(\|\Xp-\data\Xp_i\|)$, $i=1,\,\ldots,\,\data\cardinality$. Attractive
choices for $\phi$ are the polyharmonic splines (PHS),
\begin{equation*}
    \text{PHS:}\quad\phi(r) = \begin{cases}
        r^{2k} \log r, \\
        r^{2k+1},
    \end{cases}
    \quad\text{for}\ k\in\mathbb{N},
\end{equation*}
which do not require a shape parameter, unlike Gaussian or multiquadric kernels~%
\cite{Fasshauer:2007ui}. However, PHS are finitely differentiable and conditionally
positive definite; they require additional polynomial terms up to degree $k$ to guarantee
a unique interpolant. Heuristically, $\data\cardinality$ is chosen so that data sites
outnumber polynomials at least 2-to-1 to maintain reasonable conditioning. On $\sphere$,
the polynomials are spherical harmonics. We denote the polynomials by $p_k(\Xp)$,
$k=1,\,\ldots,\,\poly\cardinality$. The interpolant takes the form
\begin{equation}\label{eq:rbf-interp}
    s(\Xp)
    = \sum_{i=1}^{\data\cardinality} c_i \phi(\|\Xp-\data\Xp_i\|)
    + \sum_{k=1}^{\poly\cardinality} d_k p_k(\Xp),
\end{equation}
and exactly recovers $\psi$ at each of the data sites,
$s(\data\Xp_j) = \psi(\data\Xp_i)$, for $j = 1,\,\ldots,\,\data\cardinality$. We further
constrain the coefficients $c_i$ so that the polynomials recover polynomial data,
\begin{equation}
    \sum_{i=1}^{\data\cardinality} c_i p_k(\data\Xp_i) = 0.
    \label{eq:constraints}
\end{equation}
Collecting the values $\Phi=(\phi(\|\data\Xp_j-\data\Xp_i\|))$, $P=(p_k(\data\Xp_j))$,
$\arr{c} = (c_i)$, $\arr{d} = (d_k)$, and $\arr{\psi} = (\psi(\data\Xp_j))$, we form
the dense symmetric block system
\begin{equation}\label{eq:rbf-interp-matrix}
    \left[\begin{array}{cc}
            \Phi & P \\ P^T & 0
    \end{array}\right]\left[\begin{array}{c}
            \arr{c} \\ \arr{d}
    \end{array}\right] = \left[\begin{array}{c}
            \arr{\psi} \\ \arr{0}
    \end{array}\right],
\end{equation}
where the matrix block $0$ is the $\poly\cardinality\times\poly\cardinality$ zero matrix
and $\arr{0}$ is a vector of $\poly\cardinality$ zeros. Because $\data\sites$ is fixed,
we need only construct this matrix and compute its factors once.

For RBC and platelet parametrizations, we identify the point $\X(\theta,\,\varphi,\,t)$
on $\interface$ with the point $\Xp(\theta,\,\varphi)$ on $\sphere$. Each component of
$\X$ is a function defined on $\sphere$. By sampling $\X$ at each point in $\data\sites$,
we can approximately reconstruct the surface by interpolating each of the components. It
is clear from Equations~\eqref{eq:skalak-law}--\eqref{eq:dissip-energy} that computing
the force densities requires values of $I_1$, $I_2$, and $H$, among others. These values
are derived from the first and second derivatives of $\X$. See~\ref{sec:forces} for
details. To meet the smoothness requirements for evaluating the force densities, we use
$\phi(r) = r^7$ for RBCs and platelets, with up to $5\th$ order spherical harmonics for
RBCs and just the constant polynomial for platelets. While this does not guarantee a
unique interpolant for the platelet, the resulting system~\eqref{eq:rbf-interp-matrix} is
invertible nonetheless. The interpolants are then thrice differentiable. We next describe
the construction of the necessary discrete differential operators.

\subsection{Discrete linear surface operators}

Let $\L$ be a linear operator. In particular, we are interested in the first- and
second-order partial differential operators, $\partial/\partial\theta$,
$\partial^2/\partial\theta\partial\varphi$, \latin{etc}. We approximate $\L\psi$ by
applying $\L$ analytically to the interpolant $s$ defined in Equation~%
\eqref{eq:rbf-interp}. This is straightforward, given a parametrized metric. For
$\sphere$, this is
\begin{equation}\label{eq:sphere-metric}
    \begin{aligned}
    \|\Xp(\theta_j,\,\varphi_j) - \Xp(\theta_i,\,\varphi_i)\|
    &= \sqrt{2(1 - \cos\varphi_j\cos\varphi_i\cos(\theta_j-\theta_i) - \sin\varphi_j\sin\varphi_i)} \\
    &= \sqrt{2(1-\Xp(\theta_j,\,\varphi_j)\cdot\Xp(\theta_i,\,\varphi_i))}.
\end{aligned}
\end{equation}
However, evaluating $\L s$ at each data site involves dense operations against a
$\data\cardinality\times(\data\cardinality+\poly\cardinality)$ matrix. Depending on the
needs of the simulation, the number of data sites may be large.  In the interest of
saving memory and time for such cases, we opt instead to use fewer data sites to
reconstruct the surface, and choose a larger set of $\sample\cardinality$ \emph{sample
sites}, $\sample\sites$, at which to evaluate $\L$. We must then also consider $\L$ the
identity operator in order to obtain $\X$ at sample sites. Evaluating $\L s$ at each
sample site and collecting values $\L\Phi = (\L\phi(\|\Xp-\data\Xp_i\|)|_{\Xp=\Xpsj})$
and $\L P = (\L p_k(\Xp)|_{\Xp=\Xpsj})$, where $\Xpsj = \Xp(\theta_j,\,\varphi_j)$ for
each $(\theta_j,\,\varphi_j)\in\sample\sites$,\comment{}{$\Xpsj$ is \emph{not} a sample
site, but $(\theta_j,\,\varphi_j)\in\sample\sites$ is.} we have
\begin{equation}\label{eq:rbf-operator}
    \begin{aligned}
    \left[\begin{array}{cc}
            \L\Phi & \L P
    \end{array}\right]\left[\begin{array}{c}
            \arr{c} \\ \arr{d}
    \end{array}\right] &\hphantom{:}=
    \left[\begin{array}{cc}
            \L\Phi & \L P
    \end{array}\right]\left[\begin{array}{cc}
            \Phi & P \\ P^T & 0
    \end{array}\right]^{-1}\left[\begin{array}{c}
            \arr{\psi} \\ \arr{0}
    \end{array}\right] \\ &:=
    \left[\begin{array}{cc}
            L & \ast
    \end{array}\right]\left[\begin{array}{c}
            \arr{\psi} \\ \arr{0}
    \end{array}\right],
\end{aligned}
\end{equation}
where we have used Equation~\eqref{eq:rbf-interp-matrix} to substitute for $\arr{c}$ and
$\arr{d}$. The matrix $L$ is the discrete analogue of $\L$ applied at each sample site.
It is important to note that $L$ is completely independent of the function $\psi$; in
fact, it depends only on the functions $\phi$ and $p_k$, the fixed data sites
$\data\sites$, and the fixed sample sites $\sample\sites$.  The block marked by $\ast$ is
multiplied by zeros, and can be discarded. We compute a separate $L$ for each operator
$\L$ as a preprocessing step, and simply apply these matrices to any function that needs
to be evaluated or differentiated. It is also straightforward to generate versions of $L$
that produce derivatives at the data sites simply by replacing $\Xpsj$ with $\data\Xp_j$
in the above discussion. With these operators in hand, the force densities are readily
discretized. The quantities $I_1$, $I_2$, and $H$ in Section~\ref{sec:energy} are
calculated using local first and second derivatives. Application of the dense discrete
differential operators is performed in parallel with a parallel implementation of
\texttt{BLAS}. Lagrangian forces can therefore be computed in parallel with few thread
synchronizations.

We now have a method for computing a suitable set of points and for discretizing $\F$ for
use in~\eqref{eq:ib-spread}. To compute a force from a force density, we need to
approximate quadrature weights, or surface patch areas, for each sample site. The
following section is devoted to computing quadrature weights $\weight[j]$ for each
sample site using RBFs.

% vim: cc=90 tw=89
