\appendix
\section{Boundary error correction for staggered grids}\label{sec:boundary-correction}

Of particular interest is the Laplacian operator, which appears in every linear
solve. Consider the 1-dimensional diffusion problem for $u = u(x, t)$,
\begin{alignat}{2}
    u_t      &= \mu u_{xx} + f           &&\quad \text{for} \ x\in(0,\,1), \label{eq:1d-diff} \\
    \gamma_0 &= \alpha_0 u + \beta_0 u_x &&\quad \text{at} \ x=0, \label{eq:1d-0bcs} \\
    \gamma_1 &= \alpha_1 u - \beta_1 u_x &&\quad \text{at} \ x=1, \label{eq:1d-1bcs}
\end{alignat}
where $\alpha_m^2 + \beta_m^2 \neq 0$ for $m = 0,\,1$. Here, primes denote
differentiation with respect to $x$. We discretize the domain into cells of length
$h=1/N$ and approximate $u(x_A)$ with $u_A$ for $x_A = h(g + A-1)$, where $A$ ranges over
$\range{N}$ and $g\in(0,\,1]$ is the grid staggering. Consider the boundary at $x=0$. Let
$x_\text{g} = h(g-1)$ be a ghost point to the left of the boundary. Near the boundary, we
use the linear approximations
\begin{equation}
    u(0, t) \approx au_1 + bu_g \quad \text{and} \quad u_x(0, t) \approx a'u_1 + b'u_g.
    \label{eq:1d-approx}
\end{equation}
We require that these approximations be at least first order: $a+b=1$, $a'+b'=0$, and
$a'hg + b'h(g-1) = 1$. Dropping subscripts for $\alpha_0$, $\beta_0$, and $\gamma_0$,
and substituting into the boundary condition yields
\begin{equation}
    \gamma = \alpha(au_1 + bu_g) + \beta(a'u_1 + b'u_g) + \mathcal{O}(h).
    \label{eq:1d-bc}
\end{equation}
Given a value $u_1$ and boundary data $\gamma$, we can compute
\begin{equation}
    u_g \approx (\alpha b + \beta b')^{-1}(\gamma-(\alpha a + \beta a')u_1),
\end{equation}
when $\alpha b + \beta b' \neq 0$. In the extraordinary case that
$\alpha b + \beta b' = 0$, the boundary condition is of Robin type, with neither
$\alpha$ nor $\beta$ zero, the value at the ghost point is arbitrary, and solving
$\alpha u_1 = \gamma$ determines the value $u_1$. Assuming $\alpha b + \beta b' \neq 0$,
the standard 3-point discrete Laplacian at $x_1$ gives the approximation
\begin{equation}
    u_g - 2u_1 + u_2
    = h^2\left[(\alpha b + \beta b')^{-1}\gamma - \left(2 + (\alpha b + \beta b')^{-1}(\alpha a + \beta a')\right)u_1 + u_1\right].
    \label{eq:1d-3ptl}
\end{equation}
Replacing approximations with exact values and Taylor expanding about $x_1$ yields
\begin{equation}
    \begin{aligned}
        u_g - 2u_1 + u_2
        &= (\alpha b + \beta b')^{-1}\left(\alpha\left(u-hgu_{xx} + \sfrac12(hg)^2 u_{xx}\right) + \beta(u_x - hgu_{xx})\right) \\
        &\quad -\left(2+(\alpha b+\beta b')^{-1}(\alpha a + \beta a')\right)u + \left(u + hu_x + \sfrac12 h^2u_{xx}\right) + \mathcal{O}(h^3) \\
        &= (\alpha b + \beta b')^{-1}(\alpha(1-a-b) - \beta(a'+b'))u \\
        &\quad +(\alpha b + \beta b')^{-1}(\beta - \alpha hg + (\alpha b + \beta b')h)u_x \\
        &\quad +(\alpha b + \beta b')^{-1}\left(\sfrac12\alpha(gh)^2-\beta hg +\sfrac12(\alpha b + \beta b')h^2\right)u_{xx} + \mathcal{O}(h^3).
    \end{aligned}
\end{equation}
The coefficient of $u$ vanishes under the assumption that the approximations
\eqref{eq:1d-approx} be at least first order. We further require that the coefficient of
$u'$ be zero. That is,
\begin{equation*}
    \alpha b + \beta b' = -h^{-1}(\beta-\alpha hg).
\end{equation*}
Choosing $a=1-g$, $b=g$, $a'=h^{-1}$, and $b'=-h^{-1}$ satisfies these conditions. As a
result,
\begin{equation*}
    \alpha a + \beta a' = h^{-1}(\beta + \alpha h(1-g)).
\end{equation*}
Finally, the first possibly nonzero coefficient is that of $u_{xx}$:
\begin{equation}
    \begin{aligned}
        h^{-2}\left[u_g - 2u_1 + u_2\right]
        &= \left[\sfrac12 + g(\beta-\alpha hg)^{-1}(\beta-\sfrac12 hg)\right]u_{xx} + \mathcal{O}(h^2) \\
        &= \left[1-\sfrac12(\beta-\alpha hg)^{-1}(\beta(1-2g)-\alpha hg(1-g))\right]u_{xx} + \mathcal{O}(h^2) \\
        &:= (1-\epsilon)u_{xx} + \mathcal{O}(h^2).
    \end{aligned}
    \label{eq:lap-error}
\end{equation}
Near the boundary, when $\epsilon \neq 0$, i.e., $\beta(1-2g)-\alpha hg(1-g)\neq0$,
these approximations yield a $0^\text{th}$-order approximation to the Laplacian. Cases
where $\epsilon \neq 0$ arise naturally from using staggered grids in a domain with at
least one non-periodic dimension. In fact, for fixed $\alpha$ and $\beta$, only
$g=g^\ast(\sfrac{2\beta}{\alpha h})$ results in $\epsilon = 0$, where
\begin{equation*}
    g^\ast(r) = \begin{cases}
        \sfrac12\left(1+r+\sqrt{1+r^2}\right), & r \le 0 \\
        \sfrac12\left(1+r-\sqrt{1+r^2}\right), & r > 0.
    \end{cases}
\end{equation*}
For Neumann boundaries, $\epsilon = 0$ when $g = 0.5$; for Dirichlet boundaries, when
$g = 1$. The case for the opposing boundary is very similar: simply substitute the
correct boundary condition coefficients and data, $-h$ for $h$, and when $g\neq1$, $1-g$
for $g$.  We will use $\epsilon_0$ and $\epsilon_1$, when necessary, to distinguish the
error factor when approximating the Laplacian at the $x=0$ and $x=1$ boundaries,
respectively.

\begin{figure}[t]
\centering
\begin{tikzpicture}
    \begin{axis}[view={0}{90}, colorbar horizontal, xmin=0, xmax=0.5, ymin=0.01, ymax=0.99, ylabel=$x$, xlabel=$t$, scale only axis, width=1.95in, clip=false]
        \addplot3[surf, mesh/cols=100, shader=interp] file {dirichlet.dat};
        \node at (0.45, 0.9) {(a)};
    \end{axis}
\end{tikzpicture}
\begin{tikzpicture}
    \begin{axis}[view={0}{90}, colorbar horizontal, xmin=0, xmax=0.5, ymin=0.01, ymax=0.99, xlabel=$t$, ytick=\empty, scale only axis, width=1.95in, clip=false]
        \addplot3[surf, mesh/cols=99, shader=interp] file {neumann.dat};
        \node at (0.45, 0.9) {(b)};
    \end{axis}
\end{tikzpicture}
\caption{%
    Propagation of errors near the boundary in approximating the solution of
    $u_t = u_{xx} + 1$ on $[0,\,1]$ without correction at the boundary. Initially, $u$ is
    analytically steady: $u(x,\,0) = x(1-x)/2$. We expect no change in $u$ over time.
    (a) $u$ satisfies homogeneous Dirichlet boundary conditions. The domain is
    discretized using $h=0.01$, with points $x_i=h(i-0.5)$ for $i=1,\,\ldots,\,100$.
    (b) $u$ satisfies the Neumann boundary conditions $u_x(0,\,t)=-u_x(1,\,t)=\sfrac12$.
    The domain is discretized using $h=0.01$, with points $x_i = hi$ for
    $i=1,\,\ldots,\,99$.
}
\label{fig:error}
\end{figure}

Suppose that in approximating the solution to \eqref{eq:1d-diff}--\eqref{eq:1d-1bcs}, we
employ the Crank-Ni\-col\-son/midpoint IMEX time-stepping scheme. The discrete equations
are
\begin{equation}
    \frac{u_i^{n+1}-u_i^n}{k} = \frac{\mu}2\left(\frac{u_{i-1}^{n+1}-2u_i^{n+1}+u_{i+1}^{n+1}}{h^2} + \frac{u_{i-1}^n-2u_i^n+u_{i+1}^n}{h^2}\right) + f_i^{n+1/2},
    \label{eq:disc-1d-diff}
\end{equation}
where superscripts denote the time step and subscripts the space step. With
$\lambda=\mu k$, we rewrite this in matrix form as
\begin{equation}
    (I-\sfrac12\lambda L_h)\vec{u}^{n+1} = (I+\sfrac12\lambda L_h)\vec{u}^n + \lambda B_h\vec{\gamma}^{n+1/2} + k\vec{f}^{n+1/2},
    \label{eq:disc-1d-diff-mat}
\end{equation}
where $B_h$ is an appropriate operator that modifies the equations near the boundary with
boundary data. As we have shown, this yields an $\mathcal{O}(1)$ error near the boundary.
This error will propagate into the center of the domain at a rate dependent upon $\mu$.
Figure \ref{fig:error} illustrates this phenomenon. Consider, instead, approximating the
solution to $(1-\epsilon) u_t = \mu (1-\epsilon) u_{xx} + (1-\epsilon) f$. The solution
should be identical to that of the original equation as long as $\epsilon\neq 1$. The
Laplacian constructed above need not be modified to approximate $(1-\epsilon) u_{xx}$
near the boundary, so we simply multiply the rest of the terms by $1-\epsilon$. Define
the modified identity matrix
\begin{equation}
    \tilde{I} = \left[\begin{array}{ccccc}
            1-\epsilon_0 &   &        &   &         \\
                         & 1 &        &   &         \\
                         &   & \ddots &   &         \\
                         &   &        & 1 &         \\
                         &   &        &   & 1-\epsilon_1
            \end{array}\right].
    \label{eq:mod-ident}
\end{equation}
Now, equation \eqref{eq:disc-1d-diff-mat} becomes
\begin{equation}
    (\tilde{I}-\sfrac12\lambda L_h)\vec{u}^{n+1} = (\tilde{I}+\sfrac12\lambda L_h)\vec{u}^n + \lambda B\vec{\gamma}^{n+1/2} + k\tilde{I}\vec{f}^{n+1/2}.
    \label{eq:1d-diff-mat}
\end{equation}
This improves the error near the boundary to at least first order at the cost of one more
diagonal matrix-vector multiplication.

Alternatively, one could approximate the Laplacian near the boundary using a quadratic
interpolant. It would always give a second-order approximation, but would break symmetry
of the discrete Laplacian. On the other hand, the linear interpolants maintain symmetry,
and the coefficients obtained near the boundary are exactly those of the quadratic
interpolant, scaled by $1-\epsilon$. Thus, the modifications described above yield a
second-order approximation everywhere.

To construct higher-dimension Laplacians, we use the modified identity~%
\eqref{eq:mod-ident} in place of the normal identity in the tensor sum
\begin{equation*}
    A\oplus B = A\odot \tilde{I}_B + \tilde{I}_A\odot B,
\end{equation*}
where $A$ and $B$ are square matrices, $\tilde{I}_A$ and $\tilde{I}_B$ are modified
identites of the same size as $A$ and $B$, respectively, and $\odot$ denotes the
Kronecker tensor product. The Kronecker product of $\tilde{I}_A$ and $\tilde{I}_B$ yields
a suitable modified identity for the higher-dimensional Laplacian for constructing the
necessary Helmholtz operators and modifying right-hand sides.
