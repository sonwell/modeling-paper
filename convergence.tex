\subsection{Convergence study}

In this section, we perform a series of tests on a single perturbed RBC undergoing
relaxation. We simplify the RBC model and use only Skalak's Law. We expect the IB method
to approximate the fluid velocity at first order for thin shells, as it cannot recover
the pressure jump across the interface. We stretch the RBC by a factor of 1.1 in the $z$
direction and compress it in the $x$ direction to maintain its reference volume. We place
the cell in the center of a $16\um\times16\um\times16\um$ domain with homogeneous
Dirichlet boundary conditions in the $y$ direction and periodic boundaries elsewhere. The
fluid velocity is initially zero. The cell is then allowed to relax for $180\us$.

\begin{table}[t!]
    \centering
    \caption[Convergence of fluid velocities for relaxing RBC test]{%
Convergence of $\u$ for a sequence of grids. The refinement ratio $r$, defined as the
refinement in $h$ relative to the coarsest grid, determines the simulation parameters:
$rh = 0.8\um$ and $r\timestep = 180\ns$. Errors are computed between grids of refinement
factor $r$ and $r+1$. Values of $\u$ are sampled at $t = 180\us$ at cell centers on the
coarsest grid.
    }\label{tab:u-rbc-conv}
    \begingroup
    \setlength{\tabcolsep}{9pt}
    \renewcommand{\arraystretch}{1.5}
    \begin{tabular}{c|cc|cc}
                                                                                     \toprule
        $r$ & $L_2$ error            & order   & $L_\infty$ error       & order   \\ \midrule
        1   & $\scinot{1.74592}{-3}$ &         & $\scinot{1.59995}{-2}$ &         \\
        2   & $\scinot{9.92788}{-5}$ &         & $\scinot{6.52038}{-4}$ &         \\
        3   & $\scinot{3.65264}{-5}$ & 6.85983 & $\scinot{3.34322}{-4}$ & 1.06075 \\
        4   & $\scinot{2.31069}{-5}$ & 2.87612 & $\scinot{2.30732}{-4}$ & 1.60689 \\
        5   & $\scinot{1.65898}{-5}$ & 1.25668 & $\scinot{2.28193}{-4}$ & 0.83030 \\ \bottomrule
    \end{tabular}
    \endgroup
\end{table}

We use the 3-point kernel $\hat{\Dirac}_1$ derived by Roma \latin{et al.}~\cite{Roma:1999tx} for
spreading and interpolation and the 2-stage RK method described in Section~\ref{sec:ib}
to advance the fluid velocity. Fluid grids are chosen to have $20r$ grid points per
$16\um$ in each direction for $r$ from 1 to 6. On successive grids, we compare the fluid
velocity at cell centers in a regular grid of $20^3$ cells and surface positions at 1000
surface points. For convergence of order $p$, we expect the ratio of successive errors to
satisfy
\begin{equation*}
    \frac{\epsilon_r}{\epsilon_{r+1}} = \left|\frac{((r+1)/r)^p-1}{((r+2)/(r+1))^{-p}-1}\right|,
\end{equation*}
which we solve numerically for $p$. We compute the errors in $\X$ using discrete versions
of the $L_2$ and $L_\infty$ norms,
\begin{gather}
    \|\X(\theta,\,\varphi)\|_2^2 =
    \int\limits_{\sphere} \X(\theta,\,\varphi)\cdot\X(\theta,\,\varphi) \d\qs \quad\text{and} \\
    \|\X(\theta,\,\varphi)\|_\infty^2 =
    \max_{(\theta,\,\varphi)} \X(\theta,\,\varphi)\cdot\X(\theta,\,\varphi),
\end{gather}
respectively.

\begin{table}[t!]
    \centering
    \caption[Convergence of surface positions for relaxing RBC test]{%
Convergence of $\X$ for a sequence of grids. The refinement ratio $r$, defined as the
refinement in $h$ relative to the coarsest grid, determines the simulation parameters:
$r\timestep = 180\ns$, $n_d = 125r^2$, and $n_s = 500r^2$. Errors are computed between
grids of refinement factor $r$ and $r+1$. Values of $\X$ are sampled at $t = 180\us$ at
$N=1000$ Bauer spiral points.
    }\label{tab:x-rbc-conv}
    \begingroup
    \setlength{\tabcolsep}{9pt}
    \renewcommand{\arraystretch}{1.5}
    \begin{tabular}{c|cc|cc}
                                                                                     \toprule
        $r$ & $L_2$ error            & order   & $L_\infty$ error       & order   \\ \midrule
        1   & $\scinot{6.05447}{-3}$ &         & $\scinot{2.68249}{-3}$ &         \\
        2   & $\scinot{1.61678}{-3}$ &         & $\scinot{6.86140}{-4}$ &         \\
        3   & $\scinot{7.69150}{-4}$ & 2.74664 & $\scinot{3.30734}{-4}$ & 2.86457 \\
        4   & $\scinot{4.66203}{-4}$ & 1.89474 & $\scinot{1.86139}{-4}$ & 1.84435 \\
        5   & $\scinot{2.82578}{-4}$ & 1.46574 & $\scinot{1.18713}{-4}$ & 1.82742 \\ \bottomrule
    \end{tabular}
    \endgroup
\end{table}

Tables~\ref{tab:u-rbc-conv} and~\ref{tab:x-rbc-conv} show $L_2$ and
$L_\infty$ errors in $\u$ and $\X$ between successive grids. We observe first-order
convergence, as expected.  Satisfied with the convergence of our implementation, we
henceforth continue using the Bauer spiral to discretize RBCs and use the complete RBC
model, which we verify in the next section.
