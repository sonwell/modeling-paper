\section{Force models}\label{sec:force}

In this section, we describe the various forms of force density used in our simulations
and give analytic expressions for each. We consider four kinds of energy densities:
(damped) spring forces, tensions, dissipative forces, and Canham-Helfrich bending forces.
We combine these for the different types of cells used in our simulations.

For simplicity in describing these forces, we adopt the Einstein summation notation in
this section. A Greek letter featuring as a subscript and superscript within a term
indicates summation over ${1,\,2}$ for that letter. For example, we write the vector of
surface coordinates in terms of its components as $\qs = \q[\alpha]\e_\alpha$. We also
adopt the comma notation for partial differentiation, where subscripts following a comma
indicate partial differentiation with respect to the corresponding coordinates, e.g.,
$\phi_{,\alpha\beta} = \partial^2\phi/\partial\q[\alpha]\partial\q[\beta]$.

We begin with Hookean and damped spring forces, which are the simplest of the forces we
consider. The spring force density takes the form
\begin{equation}
    \F_\text{spring} = -k (\X - \X') - \eta(\vec{\dot{X}}-\vec{\dot{X}}').
\end{equation}
where $\X'$ is the tether location, $\vec{\dot{X}}$ is the surface velocity,
$\vec{\dot{X}}'$ is the prescribed velocity, $k$ is the spring constant, and $\eta$ is
the damping constant.  It is common practice to treat each spring as an individual, so
that the quadrature weight $\weight[A]$ is absorbed into the coefficients, and $k$ has
units of force per length and $\eta$ units of force-time per length. Implementation of
these forces requires no geometric information outside of positions and velocities.

Detailed energy models are unavailable for the endothelium, but endothelial cells are
strongly bound to the subendothelium. We take a simple approach by treating the
endothelium as a layer of stiff springs with $k = 2.5\dynpercm$,
$\eta = \scinot{1}{-7}\dynsecpercm$, and zero prescribed velocity. In Section~%
\ref{sec:whole-blood}, we compare different choices for $\X'$.

Next, we consider tension, which generates forces based on stretching or compressing
and relative direction of the surface tangent vectors. We construct the \emph{metric
tensor}, $\metric_{\alpha\beta} = \X_{,\alpha}\cdot\X_{,\beta}$, which encodes local
information about distance and area. Its inverse, denoted by $\metric^{\alpha\beta}$, is
called the dual metric.  Likewise, we construct a metric tensor for the reference
configuration, $\reference\metric_{\alpha\beta}$, and its dual. The invariants of the
Green-Lagrange strain tensor,
\begin{equation}
    \epsilon_\alpha^\beta = \frac12\left(\metric_{\alpha\mu}\reference\metric^{\mu\beta}-\Kronecker_\alpha^\beta\right),
\end{equation}
encode information about relative changes in lengths and areas. Tension energy density
function are often expressed in terms of the invariants $I_1 = 2\epsilon_\mu^\mu$ and
$I_2 = 4\epsilon + I_1$, where $\epsilon_\mu^\mu$ and $\epsilon$ are the trace and
determinant of $\epsilon_\alpha^{\smash\beta}$, respectively. $I_1$ and $I_2$ relate to
changes in lengths and areas, respectively, so that any rigid body motion of the
reference configuration results in $I_1 = I_2 = 0$. The tension energy density function,
$W(I_1,\,I_2)$, is therefore a function only of the tangents, $\X_{,\alpha}$. Of
particular interest are the Skalak Law~\cite{Skalak:1973tp}, which was developed
specifically for RBCs, and neo-Hookean tension,
\begin{align}
    W_\text{Sk} &= \frac{E}{4}\left(I_1^2 + 2I_1 - 2I_2\right) + \frac{G}4 I_2^2, \quad\text{and} \\
    W_\text{nH} &= \frac{E}{2}\left(\frac{I_1+2}{\sqrt{I_2+1}}-2\right) + \frac{G}2 \left(\sqrt{I_2+1}-1\right)^2,
\end{align}
respectively. Here, $E$ is the shear modulus, and $G$ is the bulk modulus. The resulting
tension force density is computed~\cite{Maxian:2018ek} via
\begin{equation}\label{eq:tension-force}
    \F_\text{tension} = \frac{1}{\sqrt{\reference\metric}}\left(\sqrt{\reference\metric}s^{\alpha\beta}\X_{,\beta}\right)_{,\alpha},
\end{equation}
where the second Piola-Kirchhoff stress tensor, $s^{\alpha\beta}$, is given by
\begin{equation}
    s^{\alpha\beta} = 2\frac{\partial W}{\partial I_1} \hat{g}^{\alpha\beta} + 2I_2\frac{\partial W}{\partial I_2} g^{\alpha\beta}.
\end{equation}
Because the tension force density is expressed in relation to the reference
configuration, the force is computed by multiplying by quadrature weights for the
reference configuration, which do not change over the course of a simulation.

For RBCs, deformations generate a force density according to the Skalak Law with moduli
$E_\text{rbc} = \scinot{2.5}{-3}\dynpercm$ and $G_\text{rbc} = \scinot{2.5}{-1}\dynpercm$
and reference configuration given by Omori, \latin{et al.}~\cite{Omori:2012hw},
\begin{equation*}
    \vec{\hat{X}}_\text{rbc}(\qs) = 3.91\um\left(\cos\q[1]\cos\q[2]\e_1 + \sin\q[1]\cos\q[2]\e_2 + z\left(\cos^2\q[2]\right)\sin\q[2]\e_3\right),
\end{equation*}
where $z(r) = 0.105 + r - 0.56r^2$. Platelets do not have a purpose-built constitutive
law, but are stiffer than RBCs. We use the neo-Hookean model with
$E_\text{plt} = \scinot{1}{-1}\dynpercm$, $G_\text{plt} = 1\dynpercm$, and an ellipsoid
reference configuration~\cite{Frojmovic:1982wk},
\begin{equation*}
    \vec{\hat{X}}_\text{plt}(\qs) = 1.5\um\cos\q[1]\cos\q[2]\e_1 + 1.5\um\sin\q[1]\cos\q[2]\e_2 + 0.55\um\sin\q[2]\e_3.
\end{equation*}

These cells also respond to changes in the curvature of the membrane. To compute bending
force density, we first construct the unit normal vector by taking the cross product of
the tangent vectors and normalizing:
\begin{equation}\label{eq:unit-normal}
    \n = \frac{1}{\sqrt{\metric}} (\X_{,1}\times\X_{,2}).
\end{equation}
The tensor $b_{\alpha\beta} = \n\cdot\X_{,\alpha\beta}$ encodes information about the
curvature of the surface. The principle curvatures are the eigenvalues of the
\emph{shape tensor}, $K_\alpha^\beta = b_{\alpha\mu}\metric^{\mu\beta}$, with
$2H = K_\mu^\mu$ the total curvature, and $K$ the Gaussian curvature. The Canham-Helfrich
force density takes the form~\cite{Zhongcan:1989ue}
\begin{equation}\label{eq:bending-force}
    \F_\text{CH} = -4\kappa\left(\laplacian(H-H')+2(H-H')(H^2-K+HH')\right)\n,
\end{equation}
where $\laplacian$ is the Laplace-Beltrami operator. The constants $\kappa$ is the
bending modulus in units of energy, and $H'$ is the spontaneous or preferred curvature.
%$H' = 0$ is a common choice, so that the surface attempts to become more spherical, but
%we also use $H' = \hat{H}$, the reference curvature, so that the surface prefers to be in
%its reference configuration.
We can compute $H$ and $K$ using the formulas above, but
$\laplacian H$ requires up to fourth derivatives of $\X$. In lieu of computing
higher-order derivatives, we compute $H$ and apply the discrete Laplace-Beltrami
operator.

RBCs generate a relatively weak response to changes in curvature. We use a bending
modulus of $\kappa_\text{rbc} = \scinot{2}{-12}\si{erg}$ and a preferred curvature
$H' = 0$ for RBCs. RBCs therefore tend to locally flatten their membranes. For platelets,
we use a larger bending modulus of $\kappa_\text{plt} = \scinot{2}{-11}\si{erq}$ and
a preference for its reference configuration, $H' = \hat{H}$. Together with the
neo-Hookean tension above, this maintains a fairly rigid platelet.

Finally, we consider dissipative forces, which cause the membrane to exhibit a
viscoelastic response to strain. With surface velocity $\vec{\dot{X}}$, the metric tensor
changes in time according to 
\begin{equation}
    \dot{\metric}_{\alpha\beta} = \vec{\dot{X}}_{,\alpha}\cdot\X_{,\beta} + \X_{,\alpha}\cdot\vec{\dot{X}}_{,\beta}.
\end{equation}
The dissipative force density takes the form~\cite{Rangamani:2012hi}
\begin{equation}\label{eq:dissip-force}
    \F_\text{dissip} = \frac{\nu}{\sqrt{\metric}}\left(\sqrt{\metric}\metric^{\alpha\mu}\dot{\metric}_{\mu\lambda}\metric^{\lambda\beta}\X_{,\beta}\right)_{,\alpha},
\end{equation}
where $\nu$ is the membrane viscosity. We imbue only the RBC with viscoelasticity. While
Evans and Hochmuth suggest a viscosity of approximately $\scinot{1}{-3}\dynsecpercm$~%
\cite{Evans:1976tx}, we find this to be prohibitively expensive and instead use
$\nu_\text{rbc} = \scinot{2.5}{-7}\dynsecpercm$.

Rewriting~\eqref{eq:tension-force},~\eqref{eq:dissip-force}, and the Laplace-Beltrami
operator in~\eqref{eq:bending-force} in terms of first and second derivatives with
respect to parametric variables,
\begin{equation}\label{eq:expanded-op}
    \frac1{\sqrt{\metric}}\left(\sqrt{\metric}D^\alpha_\mu \metric^{\mu\beta}\phi_{,\beta}\right)_{,\alpha}
    = \left[\left(D^\alpha_\mu \metric^{\mu\beta}\right)_{,\alpha} + \frac{\metric_{,\alpha}}{2\metric}D^\alpha_\mu \metric^{\mu\beta}\right]\phi_{,\beta}+D^\alpha_\mu \metric^{\mu\beta}\phi_{,\alpha\beta},
\end{equation}
with $D^\alpha_\mu = s^{\alpha\beta}\metric_{\beta\mu}$,
$D^\alpha_\mu = \nu \metric^{\alpha\beta}\dot{g}_{\beta\mu}$, and
$D^\alpha_\mu = \delta^\alpha_\mu$,
respectively, we find that we require only discrete first and second derivative operators
to compute a wide variety of forces. The following section describes our methods for
reconstructing cell surfaces from a set of points and discretizing the necessary
differential operators.
