\subsection{Energy Estimates}\label{sec:energy}

As mentioned in Section~\ref{sec:rbfib}, the RBF-IB method uses different sets of
Lagrangian points for spreading forces and updating immersed structures. We now track the
total energy of the combined fluid-RBC system in the relaxation test described above.
This energy is given as 
\begin{align}
    E = \int_\domain \left[\frac{\density}2\u\cdot\u + \int_\interface \Dirac(\x-\X)W(\X,\,\ldots)\d\X\right] \d\x.
\label{eq:energy}
\end{align}
where $W(\X,\,\ldots)$ is the RBC energy density function. The first term corresponds to
the energy of the fluid, and the second to the energy of the structure. Since the
relaxation test involves an initial increase in kinetic energy of the fluid followed by a
gradual dissipation of the potential energy of the RBC as it relaxes according to the
Skalak law, we expect the total energy of the system to decrease over time. We plot a
discretized version of the energy~\eqref{eq:energy} as a function of time in Figure~%
\ref{fig:energies} for refinement factors $r=5$ and $r=6$. Figure~\ref{fig:energies}
clearly shows that our method dissipates energy in the manner expected in this problem.
We also observed that the RBF-IB method was stable for representing RBCs and platelets in
whole blood simulations.

\begin{figure}[tbp]
\centering
\begin{tikzpicture}
\begin{groupplot}[
    group style={
        y descriptions at=edge left,
        group name=energy,
        group size=2 by 1
    },
    width=2.5in,
    height=2.5in,
    xmin=-10,
    xmax=190,
    ymin=9e-12,
    ymax=1.1e-9,
    ymode=log,
    log basis y=10,
    log origin=infty,
    axis x line=bottom,
    axis y line=left,
    xlabel={time ($\us$)},
    xlabel near ticks,
    ylabel near ticks,
]
\nextgroupplot[ylabel={energy ($\erg$)}]
    \addplot+[only marks, mark options={fill=tol/vibrant/blue}] coordinates {%
        (  0, 48.4709e-11)
        ( 18, 7.56723e-11)
        ( 36, 3.71642e-11)
        ( 54, 3.14419e-11)
        ( 72, 3.0141e-11)
        ( 90, 2.94874e-11)
        (108, 2.89456e-11)
        (126, 2.84339e-11)
        (144, 2.79377e-11)
        (162, 2.74534e-11)
        (180, 2.69799e-11)
    }; \label{plot:energy100}
    \node [fill=white] at (rel axis cs: 0.075, 0.95) {(a)};
\nextgroupplot
    \addplot+[only marks, mark options={fill=tol/vibrant/blue}] coordinates {%
        (  0, 4.84709e-10)
        ( 18, 7.30549e-11)
        ( 36, 3.64163e-11)
        ( 54, 3.122e-11)
        ( 72, 3.00294e-11)
        ( 90, 2.93954e-11)
        (108, 2.88559e-11)
        (126, 2.83433e-11)
        (144, 2.78457e-11)
        (162, 2.736e-11)
        (180, 2.68852e-11)
    }; \label{plot:energy120}
    \node [fill=white] at (rel axis cs: 0.075, 0.95) {(b)};
\end{groupplot}
\end{tikzpicture}
\caption{%
Energy~\eqref{eq:energy} as a function of time for the relaxing RBC test of Section~%
\ref{sec:convergence} with refinement factors (a) $r=5$ and (b) $r=6$. The refinement
factor determines the simulation parameters: spacestep $rh = 0.8\um$, timestep
$rk=180\ns$, $\data\cardinality=125r^2$, and $\sample\cardinality=500r^2$.
}\label{fig:energies}
\end{figure}

