\subsection{Surface area weights via RBF quadrature}\label{sec:rbf-quadrature}

We also use the known parametrization of $\sphere$ to compute RBF-based quadrature weights
on $\sphere$ as a preliminary step in computing quadrature weights on any surface
diffeomorphic to $\sphere$, namely, RBC and platelet membranes.

As before, consider a function $\psi:\sphere\to\R$ defined on the sphere, and a radial
kernel $K:\sphere\times\sphere\to\R$ for which
$K(\Xp_i,\,\Xp_j)\equiv\phi(\|\Xp_i-\Xp_j\|)$. We wish to find a set of quadrature
weights $\omega_j$ such that
\begin{equation}\label{eq:quad-desire}
    \int_{\sphere} \psi(\Xp) \d\Xp \approx \sum_{j=1}^{\sample\cardinality} \omega_j \psi(\Xpsj).
\end{equation}
We use a variant of the technique described by Fuselier \latin{et al.}~%
\cite{Fuselier:2013coba}.  Choosing $\psi(\Xp) = \phi(\|\Xp-\sample\Xp_i\|)$ for each
$\sample\Xp_i$,~\eqref{eq:quad-desire} becomes
\begin{equation}
    \sum_{j=1}^{\sample\cardinality} \omega_j \phi(\|\Xpsj-\sample\Xp_i\|)
    \approx \int_{\sphere}\phi(\|\Xp-\sample\Xp_i\|) \d\Xp := \L\phi|_{\Xp=\sample\Xp_i}.
    \label{eq:cond1t}
\end{equation}
However, because the spherical metric~\eqref{eq:sphere-metric} depends only on the angle
between two points, $\L\phi$ is constant over the sphere. We therefore expect the
right-hand side to be a constant, which we denote $-I_\phi$. We require further that
$\omega_j$ sum to the surface area of $\sphere$, \latin{i.e.},
\begin{equation}
    \sum_{j=1}^{\sample\cardinality} \omega_j  = 4\pi.
    \label{eq:weight-sum}
\end{equation}
Treating $I_\phi$ as an unknown scalar, we rewrite the constraints~\eqref{eq:cond1t} and%
~\eqref{eq:weight-sum} in the symmetric block linear system %for $\omega_j$. Let
\begin{equation}\label{eq:rbf-quadrature}
    \left[\begin{array}{cc}
            \Phi & \arr{1} \\ \arr{1}^T & 0
    \end{array}\right]\left[\begin{array}{cc}
            \arr{\omega} \\ I_\phi
    \end{array}\right] = \left[\begin{array}{c}
            \arr{0} \\ 4\pi
    \end{array}\right],
\end{equation}
where $\arr{\omega}$ are the unknown weights, $\Phi$ represents the evaluations of
$\phi$, and $\arr{0}$ and $\arr{1}$ are vectors of $\sample\cardinality$ zeros and ones,
respectively. $I_{\phi}$ serves as a Lagrange multiplier that enforces~%
\eqref{eq:weight-sum}, and $-I_\phi$ is a good approximation to $\L\phi$. By choosing
$\phi(r) = r$, we guarantee a unique solution and obtain weights that we observe to
converge at 3\textsuperscript{rd} order. It is possible to improve the order of the
quadrature weights by increasing the order of the PHS RBF at the potential cost of poorer
conditioning and either loss of invertibility or requiring knowledge of higher-order
moments~\cite{Fuselier:2013coba}.

Having described the computation of quadrature weights for $\sphere$, we now
describe how to compute quadrature weights for $\interface$, which has parametrization
$\X(\theta,\,\varphi)$ and Jacobian $J(\theta,\,\varphi)$. The point
$\X(\theta,\,\varphi)$ on the cell and $\Xp(\theta,\,\varphi)$ share surface coordinates
and the determinant of the Jacobian for the spherical coordinate mapping (for radius 1)
is simply $\cos\varphi$.  We use a change of variables to express the infinitesimal area
$\d\X$ as
\begin{equation}\label{eq:quad-cov}
    \d\X
    = J(\theta,\,\varphi)\d\qs
    = J(\theta,\,\varphi)\sec\varphi\d\Xp,
\end{equation}
where $\d\qs$ is an infinitesimal area in parameter space. The weights $\omega_j$ found
above are discrete analogs of $\d\Xp$ at $\Xpsj$. The discrete analog of $\d\qs$ at the
$j^\text{th}$ sample site,
\begin{equation*}
    \sigma_j=\sec\varphi_j\omega_j,
\end{equation*}
can be computed at the outset of a simulation. To avoid numerical issues, we require that
$\cos\varphi_j \neq 0$ for each sample site. This is true everywhere on $\sphere$ except
the poles, $(0,\,0,\,\pm1)$. The Bauer spiral~\eqref{eq:bauer-spiral} conveniently avoids
these points. We now arrive our weights for $\interface$. At \hi{$\X_j$}, we have
\begin{equation}
    \weight[j] = \sigma_j J_j,
\end{equation}
where $J_j = J(\theta_j,\,\varphi_j)$. Computing $\weight[j]$ given $\sigma_j$ and $J_j$
amounts to a single multiplication, which can be done trivially in parallel.

%The methods described in this section are not restricted to the sphere. In Section~%
%\ref{sec:whole-blood}, we simulate the endothelium, which is a topological torus, due to
%the periodicity of the domain. Though we do not need geometric information for the force
%model used for the endothelium, the RBF methods above are also applicable to the torus,
%and therefore the endothelium. The following section describes the necessary
%modifications for the endothelium. We use a local variation of these methods as part of
%the simulation initialization process, described in Section~\ref{sec:blood-init}.
