\subsection{Surface area weights via RBF quadrature}
We now use the known parametrization of $\sphere$ to compute RBF-based quadrature weights
for integration on any surface $\interface$ that is diffeomorphic to $\sphere$. This
proceeds in two stages: we first compute quadrature weights on $\sphere$, then use the
map from $\sphere$ to $\interface$ to obtain quadrature weights on the latter.

As before, consider a function $\psi(\Xp):\sphere \to \mathbb{R}$. Our preliminary goal
is to find a set of quadrature weights $\omega_j$ such that
\begin{equation*}
    \int\limits_{\sphere} f(\Xp) \d\Xp \approx \sum_{j=1}^{\sample\cardinality} \omega_j \psi(\sample\Xp_j).
\end{equation*}
We use a variant of the technique described by Fuselier \latin{et al.}~%
\cite{Fuselier:2013coba}.  Choosing $\psi(\Xp) = \phi(\|\Xp-\sample\Xp_i\|)$ for each
$\sample\Xp_i$ we see that
\begin{equation}
    \sum_{j=1}^{\sample\cardinality} \omega_j \phi(\|\sample\Xp_j-\sample\Xp_i\|)
    \approx \int\limits_{\sphere}\phi(\|\Xp-\sample\Xp_i\|) \d\Xp.
    \label{eq:cond1t}
\end{equation}
However, due to the homogeneity of $\sphere$, the right hand side of~\eqref{eq:cond1t} is
a constant over the sphere. We therefore expect the left hand side to be a constant,
which we denote $-I_\phi$. We can now rewrite~\eqref{eq:cond1t} as a constraint on the
quadrature weights $\omega_j$:
\begin{equation}
    \sum_{j=1}^{\sample\cardinality} \omega_j\phi(\|\sample\Xp_j-\sample\Xp_i\|)  +  I_{\phi} = 0,
    \label{eq:quadrature-weights}
\end{equation}
where $I_{\phi}$ is now treated as an unknown scalar. We require further that $\omega_j$
sum to the surface area of $\sphere$, $4\pi$, \latin{i.e.},
\begin{equation}
    \sum_{j=1}^{\sample\cardinality} \omega_j  = 4\pi.
    \label{eq:quadrature-constraint}
\end{equation}
These two constraints can again be collected into a block linear system to solve for the
weights $\omega_j$ as a preprocessing step. $I_{\phi}$ serves as a Lagrange multipler
that enforces \eqref{eq:quadrature-constraint}, and $-I_\phi$ is a good approximation to
the integral of $\phi$ over $\sphere$. By choosing $\phi(r) = r$, we guarantee a unique
solution. It is possible to improve the order of the quadrature weights by increasing the
order of the PHS at the potential cost of poorer conditioning and either loss of
invertibility or requiring knowledge of higher-order moments~\cite{Fuselier:2013coba}.

%We use the same machinery from the previous section for quadrature on the surface.
%Since we know the parametrization of $\sphere$, we can use RBFs to compute
%integration weights for the surface in a straightforward manner. Let $\L$ be the operator
%that integrates over $\sphere$ in Cartesian space. $\sphere$ is homogeneous,
%which implies invariance under rotation; $\L\phi(\|\Xp-\Xp'\|)$, for fixed $\Xp'$, is
%independent of $\Xp'$ and is therefore constant. We seek integration weights $\omega^A$
%on $\sphere$ for each of the sample sites such that
%\begin{equation}
%    \omega^A\phi(\|\sample\Xp_A-\sample\Xp_B\|) = I_\phi \approx \L\phi
%\end{equation}
%for constant $I_\phi$. Here, the subscript $\phi$ is a reference to the basic function.
%We require further that $\omega^A$ sum to the surface area of $\sphere$, $4\pi$.
%Treating $I_\phi$ as an unknown and letting $p(\Xp) = 1$, we rewrite these requirements
%as
%\begin{equation}
%    \omega^A\phi\left(\left\|\sample\Xp_A-\sample\Xp_B\right\|\right) - I_\phi p(\sample\Xp_B) = 0
%    \label{eq:quadrature-weights}
%\end{equation}
%restricted to
%\begin{equation}
%    \omega^A p(\sample\Xp_A) = 4\pi,
%    \label{eq:quadrature-constraint}
%\end{equation}
%which is similar in form for equations~\eqref{eq:interpolant}--\eqref{eq:constraints}
%with only a constant polynomial. By choosing $\phi(r) = r$, we guarantee a unique
%solution. It is possible to improve the order of the quadrature weights by increasing the
%order of the PHS at the potential cost of poorer conditioning and either loss of
%invertibility or requiring knowledge of higher-order moments, $\L p_M$, for each $p_M$
%needed to guarantee invertibility~\cite{Fuselier:2013coba}.

%For each set of quasi-uniform points on $\sphere$ we
%tested, the system is invertible for $\phi(r)=r^{2k+1}$ for $1 \le k \le 4$, but quality
%of the quadrature weights deteriorates with higher-order PHS, becoming unusable around
%$k=4$. Poor conditioning typically causes a non-constant approximation $I_\phi$ when
%higher-order spherical harmonics are included.

Having described the computation of quadrature weights for $\sphere$, we now
describe how to compute quadrature weights for $\interface$, which has parametrization
$\X(\qs)$. The Jacobian $J$ for $\X$ satisfies
%With quadrature weights for $\sphere$ at hand, we aim to compute weights for
%$\interface$, which has parametrization $\X(\qs)$. The Jacobian $J$ for $\X$ satisfies
$J^2 = \metric$. The Jacobian for $\Xp$ is known; $\tilde{J} = \cos\q[2]$. We can use a
change of variables to express the infinitesimal area $\d\X$ as
\begin{equation}\label{eq:quad-cov}
    \d\X
    = J\d\qs
    = (J/\tilde{J})\d\Xp,
\end{equation}
where $\d\qs$ is an infinitesimal area in parameter space. The weights $\omega_j$ found
above are discrete analogues of $\d\Xp$. Analytically evaluating $\tilde{J}$ at
$\sample\Xp_j$ yields $\tilde{J}_j$, and the discrete analogue of $\d\qs$ at $\qs_j$,
\begin{equation*}
    \sigma_j=\omega_j/\tilde{J}_j,
\end{equation*}
can be computed at the outset of a simulation. To avoid numerical issues, we require that
$\tilde{J}_j \neq 0$ for each sample site. This is true everywhere on $\sphere$ except
the poles, $(0,\,0,\,\pm1)$. The Bauer spiral~\eqref{eq:bauer-spiral} conveniently avoids
these points. We now arrive our weights for $\interface$. At $\X_A$, we have
\begin{equation}
    \weight[j] = \sigma_j J_j.
\end{equation}
$J_j$ is a byproduct of computing many of the force densities introduced in Section~%
\ref{sec:force}. Computing $\weight[j]$, given $\sigma_j$ and $J_j$, amounts to a single
multiplication, which can be done trivially in parallel. We observe further that $1/J$
or its reference configuration counterpart is a prefactor of the operator the computes
tension force density, dissipative force density, and the Laplace-Beltrami operator~%
\eqref{eq:expanded-op}, and the denominator of the unit normal~\eqref{eq:unit-normal}. We
can therefore forego the additional multiplication of the quadrature weights and the
division in the force density computation.

%We now have a framework in which we can compute discrete forces for certain surfaces. The
%next section gives specific details for the formulas in Sections~\ref{sec:force} and~%
%\ref{sec:surfaces} to give a complete picture of the models for RBCs, platelets, and the
%endothelium.
