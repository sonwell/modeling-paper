\section{Introduction}

Blood is a complex mixture of cellular and fluid-phase components, most notably composed of red
blood cells (RBCs) and platelets suspended in plasma. RBCs are primarily the basis for
transporting oxygen throughout the body. Platelets, meanwhile, play a key role in the
maintenance of the vasculature. Blood flows under pressure through vessels, which vary in
diameter between a few centimeters in the aorta to a few microns in the capillaries.
Healthy blood vessels are lined by a single layer of endothelial cells, called the
endothelium.  Interactions between platelets and RBCs, and between platelets and the
endothelium, are important for a platelet's function, but these interactions are
typically studied in isolation.

At rest, RBCs are biconcave disk-shaped cells approximately $8\um$ in diameter. The
volume fraction occupied by RBCs, or \term{hematocrit}, ranges from approximately 36\% to
45\% in healthy humans. In order to deliver oxygen throughout the body, RBCs must be
extremely flexible, as some vessels are smaller than the cell itself. Due to the wide
variety of shapes exhibited by RBCs, their mechanical properties were studied intently
during the 1970s and 80s. Canham theorized that the biconcave disk shape minimizes
bending energy~\cite{Canham:1970wx}. Skalak \latin{et al.} devised a purpose-built
constitutive law to describe the tension of RBC membranes~\cite{Skalak:1973tp}. Under the
assumption of a viscoelastic response, Evans \& Hochmuth estimated the membrane
viscosity~\cite{Evans:1976tx}. Mohandas \& Evans gave estimates of the shear, bulk, and
bending moduli~\cite{Mohandas:1994tg}, which have guided RBC models ever since~%
\cite{Pozrikidis:2003ft,Fai:2013do}.

Platelets in their inactive state are ellipsoidal disks, approximately 3--$4\um$ by
$1\um$ in size. They are much more rigid than RBCs due to their actin and microtubule-based
cytoskeletons, a property afforded to the platelet by its small size. Platelets are also
much less numerous, with 10--20 RBCs per platelet. Less is known about the mechanical
properties of platelets. Models range from perfectly rigid ellipsoids~\cite{Wang:2013gs}
to systems of springs with~\cite{Erickson:2010ep,Skorczewski:2013jn} or without~%
\cite{Wu:2014gt} a preferred curvature. One study estimates the shear modulus and
viscosity for platelets~\cite{Haga:1998wa}, but models tend to use a higher shear modulus
than estimated and neglect viscous effects altogether.

RBCs tend to move towards the center of a blood vessel, and in doing so may encounter
platelets, but the relative size and deformability of the RBC means a platelet is ejected
from the RBC's path, ultimately pushing the platelet into an RBC-free layer along the
vessel wall. This process, called margination, affects platelets and leukocytes (white
blood cells) alike, and is the focus of many studies~\cite{Freund:2007kx,Erickson:2010ep,
Erickson:2011cf,Zhao:2011do,Kumar:2011dd,Zhao:2012ggba,Fedosov:2012dy,Kumar:2012ie,
Fedosov:2013ul,Muller:2014is,Fedosov:2014bs, Vahidkhah:2014hy,Vahidkhah:2015ch,
Mehrabadi:2016fn}. From their marginated positions, platelets survey the vessel wall for
injury. Injury sites expose proteins, e.g., collagen and von Willebrand factor (vWF), to which
platelets can bind and become activated. Platelet contact with the injured wall is the
essential first step. This, in turn, leads the platelet to bind to the injury site and
release its own chemical signals to recruit further platelets, which eventually results
in the formation of a thrombus. All of this occurs in flowing blood, which sweeps these
chemical signals downstream. While mechanisms for platelet activation have been proposed
for low and pathologically high shear rates, the case of physiologically high shear rates
is undecided~\cite{Fogelson:2015fb}.

Models of platelet motion over a thrombus indicate that there are stagnation zones
immediately upstream and downstream of the thrombus, where the fluid velocity is very
slow, even when the thrombus protrudes only a few microns from the vessel wall~%
\cite{Skorczewski:2013jn,Wang:2013gs}. Platelets that enter these regions spend a
tend to spend time. The portion of an endothelial cell containing its nucleus also protrudes into
the vessel approximately $1\um$. Moreover, the endothelial bumps are roughly periodic. If
the endothelium creates a stagnation zone, the trailing zone from one protrusion might
lead into the leading zone of the subsequent protrusion. This may allow for the
sequestration of platelets or chemical signals. However, typical models of platelet-wall
interaction model the endothelium as a flat surface~\cite{Wu:2014gt,Vahidkhah:2015ch}.

The goal of this article is therefore to conduct 3D simulations of whole blood, incorporating
red blood cell, platelet, and endothelium interactions. Our model treats platelets and red blood cells
as discrete elastic objects immersed in and interacting with blood (which is modeled as an incompressible
Newtonian fluid). We use this model to compare the flow of
whole blood across bumpy and flat walls and characterize the behavior and interactions of platelets. To
simulate this model, we develop a cohesive numerical framework comprising a fluid-structure interaction method, a meshless parametric modeling technique for reconstructing cell
surfaces from point clouds and computing elastic force densities on these surfaces, and a meshless quadrature
scheme that enables numerical integration of force densities on these surfaces as well.

We use the immersed boundary (IB) method for fluid-structure interaction. Originally developed by Peskin to study the flow of blood around heart valves~\cite{Peskin:1972wa},
it has since been used to simulate, among numerous other applications, vibrations in the
inner ear~\cite{BeyerJr:1990tb}, the opening of a porous parachute~\cite{Kim:2006ku}, and
sperm motility~\cite{Dillon:2011cu}, and has generated numerous related methods. The IB method remains popular for modeling fluid-%
structure interaction because of its simplicity and ease of use, and involves maintaining an Eulerian description of the fluid
and a purely Lagrangian description of all immersed elastic structures.

For parametric modeling and force density computations on these immersed elastic structures, we utilize meshless interpolation
based on radial basis functions (RBFs), which have been used for generating differentiation matrices
for the solution of PDEs~\cite{Fasshauer:2007ui}, surface reconstruction~%
\cite{Hardy:1971tb,Carr:2001tb, Shankar:2013ki,SFKSISC2018}, and in the context of regularized Stokeslets to
represent interfaces and approximate their geometries~\cite{Olson:2015ja}. More relevantly, RBFs have been used
in the context of the IB method to reconstruct platelet surfaces from point clouds and to compute Lagrangian force densities in 2D simulations~\cite{Shankar:2015km}. However,
this RBF-IB method has yet to be applied to surface reconstruction and force density calculation in 3D simulations. Further, the 2D version of the RBF-IB method required
tuning in the RBF representation to achieve stability. In this work, we present the first extension of the RBF-IB method to the simulation of whole blood in 3D geometries.  Due to the use of the meshless high-order accurate RBF-based representation, we are able to represent the constituent cells within whole blood as point clouds with a relatively
small cardinality. Further, we eliminate the aforementioned tuning parameter using a recently-developed parameter-free RBF representation~\cite{SFKSISC2018}.

The remainder of this paper is organized as follows. We begin with an overview of the IB method in Section~\ref{sec:ib}. We then describe our method
for solving the incompressible Navier-Stokes equations in Section~\ref{sec:ins}. We
describe the energy models used for each type of cell in Section~\ref{sec:energy} and
Section~\ref{sec:rbfs} details our methods for discretizing the cells using RBFs. Our
results are presented in Section~\ref{sec:results}.  Finally, we discuss the implications
of our findings in Section~\ref{sec:conclusion}.

% vim: cc=90 tw=89
%Blood is a complex mixture of cellular and fluid-phase components, most notably composed of red
%blood cells (RBCs) and platelets suspended in plasma. RBCs are primarily the basis for
%transporting oxygen throughout the body. Platelets, meanwhile, play a key role in the
%maintenance of the vasculature. Blood flows under pressure through vessels, which vary in
%diameter between a few centimeters in the aorta to a few microns in the capillaries.
%Healthy blood vessels are lined by a single layer of endothelial cells, called the
%endothelium.  Interactions between platelets and RBCs, and between platelets and the
%endothelium, are important for a platelet's function, but these interactions are
%typically studied in isolation.
%
%At rest, RBCs are biconcave disk-shaped cells approximately $8\um$ in diameter. The
%volume fraction occupied by RBCs, or \term{hematocrit}, ranges from approximately 36\% to
%45\% in healthy humans. In order to deliver oxygen throughout the body, RBCs must be
%extremely flexible, as some vessels are smaller than the cell itself. Due to the wide
%variety of shapes exhibited by RBCs, their mechanical properties were studied intently
%during the 1970s and 80s. Canham theorized that the biconcave disk shape minimizes
%bending energy~\cite{Canham:1970wx}. Skalak \latin{et al.} devised a purpose-built
%constitutive law to describe the tension of RBC membranes~\cite{Skalak:1973tp}. Under the
%assumption of a viscoelastic response, Evans \& Hochmuth estimated the membrane
%viscosity~\cite{Evans:1976tx}. Mohandas \& Evans gave estimates of the shear, bulk, and
%bending moduli~\cite{Mohandas:1994tg}, which have guided RBC models ever since~%
%\cite{Pozrikidis:2003ft,Fai:2013do}.
%
%Platelets in their inactive state are ellipsoidal disks, approximately 3--$4\um$ by
%$1\um$ in size. They are much more rigid than RBCs due to its actin and microtubule-based
%cytoskeleton, a property afforded to the platelet by its small size. Platelets are also
%much less numerous, with 10--20 RBCs per platelet. Less is known about the mechanical
%properties of platelets. Models range from perfectly rigid ellipsoids~\cite{Wang:2013gs}
%to systems of springs with~\cite{Erickson:2010ep,Skorczewski:2013jn} or without~%
%\cite{Wu:2014gt} a preferred curvature. One study estimates the shear modulus and
%viscosity for platelets~\cite{Haga:1998wa}, but models tend to use a higher shear modulus
%than estimated and neglect viscous effects altogether.
%
%RBCs tend to move towards the center of a blood vessel, and in doing so may encounter
%platelets, but the relative size and deformability of the RBC means a platelet is ejected
%from the RBC's path, ultimately pushing the platelet into an RBC-free layer along the
%vessel wall. This is the margination process; it affects platelets and leukocytes (white
%blood cells) alike, and is the focus of many studies~\cite{Freund:2007kx,Erickson:2010ep,
%Erickson:2011cf,Zhao:2011do,Kumar:2011dd,Zhao:2012ggba,Fedosov:2012dy,Kumar:2012ie,
%Fedosov:2013ul,Muller:2014is,Fedosov:2014bs, Vahidkhah:2014hy,Vahidkhah:2015ch,
%Mehrabadi:2016fn}. From their marginated positions, platelets survey the vessel wall for
%injury. Injury sites expose proteins, e.g., collagen and von Willebrand factor, to which
%platelets can bind and become activated. Platelet contact with the injured wall is the
%essential first step. This, in turn, leads the platelet to bind to the injury site and
%release its own chemical signals to recruit further platelets, which eventually results
%in the formation of a thrombus. All of this occurs in flowing blood, which sweeps these
%chemical signals downstream. While mechanisms for platelet activation have been proposed
%for low and pathologically high shear rates, the case of physiologically high shear rates
%is undecided~\cite{Fogelson:2015fb}.
%
%Models of platelet motion over a thrombus indicate that there are stagnation zones
%immediately upstream and downstream of the thrombus, where the fluid velocity is very
%slow, even when the thrombus protrudes only a few microns from the vessel wall~%
%\cite{Skorczewski:2013jn,Wang:2013gs}. Platelets that enter these regions spend a
%disproportionate amount of time there. The endothelial cell nucleus also protrudes into
%the vessel approximately $1\um$. Moreover, the endothelial bumps are roughly periodic. If
%the endothelium creates a stagnation zone, the trailing zone from one protrusion might
%lead into the leading zone of the subsequent protrusion. This may allow for the
%sequestration of platelets or chemical signals. However, typical models of platelet-wall
%interaction model the endothelium as a flat surface~\cite{Wu:2014gt,Vahidkhah:2015ch}.
%
%To simulate whole blood, we use the immersed boundary (IB) method. The IB method was
%developed by Peskin to study the flow of blood around heart valves~\cite{Peskin:1972wa}.
%It has since been used to simulate, among numerous other applications, vibrations in the
%inner ear~\cite{BeyerJr:1990tb}, the opening of a porous parachute~\cite{Kim:2006ku}, and
%sperm motility~\cite{Dillon:2011cu}, and has generated numerous related methods. The use
%of radial basis function (RBF)-based methods to represent immersed structures within the
%IB method has been applied to flow around an aggregate of platelets, and bears the
%moniker RBF-IB~\cite{Shankar:2015km}. The IB method remains popular for modeling fluid-%
%structure interaction because of its simplicity. By treating the immersed structures as
%an extension of the fluid, an arbitrary parcel is assumed to behave like a fluid, so its
%motion is governed by the Navier-Stokes equations, driven, perhaps, by local body forces
%through which the immersed structures influence the fluid motion.
%
%In addition to the IB method, RBFs have been used for generating differentiation matrices
%for the solution of PDEs~\cite{Fasshauer:2007ui}, surface reconstruction~%
%\cite{Hardy:1971tb,Carr:2001tb}, and in the context of regularized Stokeslets to
%represent interfaces and approximate their geometries~\cite{Olson:2015ja}.
%
%In this work, we validate the RBF-IB methodology for the simulation of whole blood. We
%use it to compare the flow of whole blood across bumpy and flat walls and characterize
%the behavior and interactions of platelets. This paper proceeds incrementally, beginning
%with an overview of the IB method in Section~\ref{sec:ib}. We then describe our method
%for solving the incompressible Navier-Stokes equations in Section~\ref{sec:ins}. We
%describe the energy models used for each type of cell in Section~\ref{sec:energy} and
%Section~\ref{sec:rbfs} details our methods for discretizing the cells using RBFs. Our
%results are presented in Section~\ref{sec:results}.  Finally, we discuss the implications
%of our findings in Section~\ref{sec:conclusion}.

