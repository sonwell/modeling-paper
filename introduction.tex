\section{Introduction}

Whole blood consists of red blood cells (RBCs), platelets, and leukocytes in blood
plasma. In a healthy individual, RBCs account for 36--45\% of blood volume. They take on
a biconcave disc shape with diameter of about $8\um$ at rest, but are highly deformable
in order to pass through capillaries, which have diameters as small as $3$--$4\um$.
Because of their importance to life and the wide variety of behaviors RBCs display, they
make an interesting model system. The current RBC force model was developed and refined
in the 1970s. Canham~\cite{Canham:1970wx} proposed that the biconcave shape arises from
minimizing a bending energy. Skalak \latin{et al}.~\cite{Skalak:1973tp} developed a
tensile constitutive law for RBCs. Evans and Hochmuth~\cite{Evans:1976tx} estimated the
RBC membrane viscosity, which gives the membrane a viscoelastic response to strain.

\noindent Who did RBC stuff; things they did
\begin{itemize}[noitemsep, topsep=0pt]
    \item 3D BIM (quadratic elements) RBC with viscosity ratio 1~\cite{Pozrikidis:1998tx}
    \item 3D BIM RBC with non-unity viscosity ratio~\cite{Pozrikidis:2003ft}
    \item 3D IBFE (quadratic elements) RBC with Skalak Law and non-unity viscosity ratio~\cite{Le:2010im}
    \item 3D SDPD network-of-springs RBC with dissipative force~\cite{Fedosov:2010bc}
    \item 3D BIM RBC with Skalak Law~\cite{Omori:2012hw}
    \item 3D IB-LBM RBC~\cite{Xu:2013kk}
    \item Thomas's stuff~\cite{Fai:2013do}
    \item 3D DPD 2-component RBC model~\cite{Li:2014hva}
\end{itemize}

RBCs
outnumber platelets by a factor of approximately 20 and outnumber leukocytes by a factor
of approximately 600. Blood circulates around the body through a network of blood
vessels. A layer of endothelial cells line the blood vessels. {\XXX} There are many
simulations of blood flow at the cellular level {\XXX}. These simulations idealize the
geometry of the vessels as a flat tube. However, the endothelial cell's nucleus creates a
protrusion into the lumen of the vessel. The protrusions have prominence of approximately
$1\um$---roughly the minor axis length of the platelet.

IB, Peskin~\cite{Peskin:1972wa,Peskin:1977wza,Peskin:2002go}, blood flow through heart, yadda, yadda.

Radial basis functions (RBFs) are quickly becoming a popular tool for approximation in
a variety of settings {\XXX}. Our interest in RBFs lies in their ability to reconstruct
surfaces using scattered nodes. We represent RBCs, platelets, and the endothelium with a
set of scattered nodes, and use RBFs to reconstruct their surfaces. Within the IB
framework, this is known as the RBF-IB method~\cite{Shankar:2015km}. A smooth surface
approximant allows for the use of techniques from differential geometry to compute
surface forces~\cite{Maxian:2018ek}.


\noindent RBF stuff: quadrature~\cite{Fuselier:2013coba,Maxian:2018ek}, geometry~\cite{Shankar:2013ki,Olson:2015ja}

\noindent Whole blood types of things
\begin{itemize}[noitemsep, topsep=0pt]
    \item 3D RBC effect on platelet deposition~\cite{Wang:2013gs}
    \item 3D Platelet/microparticle margination~\cite{Vahidkhah:2014hy,Vahidkhah:2015ch}
    \item IB-LBM Platelet Ad/cohesion~\cite{Wu:2014gt}
    \item RBC-induced Leukocyte margination:
        \begin{itemize}[noitemsep, topsep=0pt]
            \item LBM simulation of 2D rigid RBCs + WBCs~\cite{Sun:2003cr}
            \item BIM simulation of 2D RBCs + WBCs~\cite{Freund:2007kx}
            \item Lindsay's stuff~\cite{Erickson:2010ep,Erickson:2011cf,Skorczewski:2013jn}
            \item SDPD simulation of 2D RBCs + WBCs~\cite{Fedosov:2012dy}
            \item SDPD simulation of 3D RBCs + WBCs~\cite{Fedosov:2014bs}
        \end{itemize}
    \item lattice boltzmann RBC: sentence on the method + cite cite cite
\end{itemize}

%This is an extension of their work.

\noindent \textcolor{red}{Placeholder for clotting paragraph}

Recent advances in parallelizing the IB method~\cite{Kassen:2020hj} allow us to
perform our simulations entirely on the GPU. Sections~\ref{sec:discretization} and~
\ref{sec:surfaces} detail our approaches to parallelizing the fluid equations and cell
forces, respectively.

The remainder of the paper proceeds as follows: Section~\ref{sec:ib} gives an overview of
the IB method; Section~\ref{sec:discretization} deals with the discretization and
solution of the incompressible Navier-Stokes equations; Section~\ref{sec:force} lists the
types of forces used in our cell models; Section~\ref{sec:surfaces} details the RBF
methods for surface reconstruction and quadrature; we report the results of our
simulations in Section~\ref{sec:results}; and, finally, we discuss the results and give
future directions in Section~\ref{sec:conclusion}.
